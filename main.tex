\documentclass[10pt,journal,compsoc]{IEEEtran}

% *** CITATION PACKAGES ***
%
\ifCLASSOPTIONcompsoc
  % IEEE Computer Society needs nocompress option
  % requires cite.sty v4.0 or later (November 2003)
  \usepackage[nocompress]{cite}
\else
  % normal IEEE
  \usepackage{cite}
\fi

\usepackage[numbers]{natbib}

\usepackage{cite}
\usepackage{amsmath,amssymb,amsfonts}
\usepackage{algorithmic}
\usepackage{graphicx}
\usepackage{textcomp}
\usepackage{xcolor}
\usepackage[framemethod=TikZ]{mdframed}
\usepackage{multirow}
\usepackage{array}
\usepackage{lipsum}
% \usepackage{subfigure}
\usepackage{caption}
\usepackage{subcaption}
\usepackage{hyperref}
\usepackage{longtable}

\newcommand{\specialcell}[2][c]{%
  \begin{tabular}[#1]{@{}c@{}}#2\end{tabular}}

\usepackage{xspace}
\newcommand{\instance}{{\em CTO}\xspace}
\newcommand{\inconsistent}{{\em IoPV}\xspace}

\newcommand{\ian}[1]{\textcolor{red}{{\it [Ian says: #1]}}}
\newcommand{\bram}[1]{\textcolor{orange}{{\it [Bram says: #1]}}}
\newcommand{\heng}[1]{\textcolor{blue}{{\it [Heng says: #1]}}}
\newcommand{\med}[1]{\textcolor{cyan}{{\it [Mohammed says: #1]}}}
\newcommand{\jinfu}[1]{\textcolor{purple}{{\it [Jinfu says: #1]}}}
% \newcommand{\heng}[1]{\textcolor{green}{{\it [Heng says: #1]}}}

\def\BibTeX{{\rm B\kern-.05em{\sc i\kern-.025em b}\kern-.08em
    T\kern-.1667em\lower.7ex\hbox{E}\kern-.125emX}}
    
    
\usepackage[most]{tcolorbox}

\hypersetup{
  colorlinks,
  citecolor=blue,
  linkcolor=red,
  urlcolor=purple}

\definecolor{custom-gray}{cmyk}{0, 0, 0, 0.7, 1.00}
\newtcbtheorem[no counter]{Summary}{\hskip-0.97em}{enhanced,drop shadow={black!50!white},
  coltitle=white,
  top=0.15in,
  attach boxed title to top left=
  {xshift=1.5em,yshift=-\tcboxedtitleheight/2},
  boxed title style={size=small,colback=custom-gray}
}{summary}

\newcommand{\PQI}{How common are \inconsistent issues? }
\newcommand{\PQII}{How difficult is it to manually identify \inconsistent issues?}

\newcommand{\RQI}{What is the impact of configuration on performance regression?}
\newcommand{\RQII}{How accurately can we predict \inconsistent issues? }
\newcommand{\RQIII}{What are the most important metrics for predicting \inconsistent issues? }


\usepackage{booktabs} % For formal tables

\usepackage{flushend} % makes the last page balanced

\usepackage{balance}

\usepackage{enumitem}
\newlist{steps}{enumerate}{1}
\setlist[steps, 1]{label = Step \arabic*:}

% code listing
\usepackage{listings}
%\usepackage{xcolor}
\definecolor{dkgreen}{rgb}{0,0.6,0}
\definecolor{gray}{rgb}{0.5,0.5,0.5}
\definecolor{mauve}{rgb}{0.58,0,0.82}
\lstset{frame=tb,
	language=Java,
	aboveskip=3mm,
	belowskip=3mm,
	showstringspaces=false,
	captionpos=b,
	columns=flexible,
	%columns=fullflexible
	%basicstyle={\small\ttfamily},
	basicstyle=\linespread{1.1}\footnotesize, 
	numbers=none,
	numberstyle=\tiny\color{gray},
	keywordstyle=\color{blue},
	commentstyle=\color{dkgreen},
	stringstyle=\color{mauve},
	breaklines=true,
	breakatwhitespace=true,
	tabsize=3,
	escapeinside={(*}{*)}
}
\lstdefinestyle{interfaces}{
	float=tp,
	floatplacement=tbp
}

\newcommand{\specialcell}[2][c]{%
	\begin{tabular}[#1]{@{}c@{}}#2\end{tabular}}
\newcommand{\specialleft}[2][l]{%
	\begin{tabular}[#1]{@{}l@{}}#2\end{tabular}}
\newcommand{\specialright}[2][r]{%
	\begin{tabular}[#1]{@{}r@{}}#2\end{tabular}}
	
\makeatletter
\newcommand\footnoteref[1]{\protected@xdef\@thefnmark{\ref{#1}}\@footnotemark}
\makeatother

\usepackage{url}
\usepackage[flushleft]{threeparttable}

\begin{document}

\title{An Empirical Study on the Inconsistent Option Performance Variation}

\author{
Jinfu Chen, Mohammed Sayagh, Heng Li, Bram Adams, and Weiyi Shang
\IEEEcompsocitemizethanks{
\IEEEcompsocthanksitem Jinfu Chen and Weiyi Shang are with the Department of Computer Science and Software Engineering, Concordia University, Canada. 
\protect\\
E-mail: \{fu\_chen, shang\}@encs.concordia.ca
\IEEEcompsocthanksitem Mohammed Sayagh is with Ecole de Technologie Superieur - Quebec University, Canada.
\protect\\
E-mail: mohammed.sayagh@etsmtl.ca
\IEEEcompsocthanksitem Bram Adams is with School of Computing, Queen’s University, Canada.
\protect\\
E-mail: bram.adams@queensu.ca
\IEEEcompsocthanksitem Heng Li is with the Département de génie informatique et génie logiciel, Polytechnique Montreal, Canada. 
\protect\\
E-mail: heng.li@polymtl.ca

}}


\IEEEtitleabstractindextext{%
\begin{abstract}

\input{tex/abstract}
\end{abstract}

\begin{IEEEkeywords}
Software Performance, Performance Evolution, Highly configurable software systems
\end{IEEEkeywords}}

\maketitle

\IEEEdisplaynontitleabstractindextext
\IEEEpeerreviewmaketitle


\section{Introduction}
\label{sec:intro}
Software systems are increasingly evolving in their size and complexity.
Large-scale software systems, such as Amazon, Google, and Facebook tend to deal with a high operation workload (e.g., millions of active users), which is complicated by the complexity and the continuous evolution of such systems. Such systems usually report more performance bugs than feature-related ones~\cite{weyuker2000experience}. As such, performance assurance activities are an essential part of the release cycle of such large-scale software systems.

On the other hand, modern large-scale software systems tend to have a large number of configuration options, which can hide performance issues. %For example, Hadoop has over 365 available configuration options~\heng{this is repeated later, may be removed}. 
These options are used to customize the behaviour of a software system without changing its source code. Although these options add flexibility to a software system, they make testing a software performance a challenging task. For example, in theory, one has to run $2^
{10}$ tests for a software system with just 10 boolean configuration options, while a highly configurable software system such as \emph{Hadoop} can have as many as 365 available options~\cite{tse}. While there are constraints between configuration options, bringing down the total number of configurations in practice, this still amounts to a too large set of configurations to test exhaustively, especially for (long-running) performance tests.
 
%Therefore, 
A large body of research proposed and evaluated approaches that detect performance issues~\cite{Nguyen:2012:ADP,nguyen2011automated,Nguyen:2014:ICS,foo2010mining,DBLP:conf/icse/FooJAHZF15}. Prior studies also proposed approaches to test the performance of highly configurable software systems~\cite{DBLP:journals/dt/SaxenaFHMYM00,wu2010performance,DBLP:journals/ese/HalinNADPB19}. However, existing approaches aimed at a late stage of the software release cycle, i.e., after the build and deployment of new releases. Nevertheless, identifying performance issues at earlier stages, especially at the development stage, can minimize the amount of resources required for identifying and fixing a performance regression. In fact, going through the whole build, testing, and packaging process to find the root cause of a performance regression is time-consuming. %\bram{is our approach able to achieve this early testing?} \jinfu{Yes, our prediction model can predict whether a CTO manifests performance variation at the commit level.}


% \begin{figure*}[tbh]
% 	\centering
% 		\includegraphics[width=.9\textwidth]{Figures/definition.pdf}
% 		\vspace{-90mm}
% 	\caption{The definition of \inconsistent and how different is it from traditional way of comparing the performance of two values for the same configuration option.\med{todo: label red with v1 and blue with v2}} 
% 	\label{fig:description} 
% \end{figure*}

\begin{figure}[t]
	\centering
        \begin{subfigure}{0.22\textwidth}
                \includegraphics[width=\linewidth]{Figures/background-a.pdf}
                \caption{}
                % \caption{Approaches that donot consider the historical evaluation}
                \label{fig:description-a}
        \end{subfigure}%
        \begin{subfigure}{0.22\textwidth}
                \includegraphics[width=\linewidth]{Figures/background-b.pdf}
                \caption{}
                % \caption{An option with an inconsistent performance variation (a-b).}
                \label{fig:description-b}
        \end{subfigure}
        \begin{subfigure}{0.22\textwidth}
                \includegraphics[width=\linewidth]{Figures/background-c.pdf}
                \caption{}
                % \caption{An option with a consistent performance variation (a-b).}
                \label{fig:description-c}
        \end{subfigure}%
        \begin{subfigure}{0.22\textwidth}
                \includegraphics[width=\linewidth]{Figures/background-d.pdf}
                \caption{}
                % \caption{An option with a consistent performance variation (a-b).}
                \label{fig:description-d}
        \end{subfigure}%
	\caption{%\bram{would the figure be easier to interpret if the long blue/red lines also would be just half the length, then put the a/b vertical areas on the border between yellow and green?} \bram{also, would be more logical to first show case c (expected behaviour), then d and b who show variations both for aggravated and improved performance} 
	The definition of \inconsistent and how different it is from the traditional way of comparing the performance of two values for the same configuration option: (a) Approaches that do not consider the historical evaluation, (b) An option with a consistent performance variation (a=b), (c) An option with an inconsistent performance variation (a!=b), and (d) An option with an inconsistent performance variation (a!=b). V1 and V2 are two different values of the same configuration option. C1 and C2 are two revisions. A smaller performance metric value (e.g., CPU usage) indicates a better performance.
	%\heng{(a) ...; (b) ...; (c) ...; (d).... Then explain what is V1, V2, and what is C1, C2. This is to make the figure self-explainable}. 
	%\heng{Make the longer lines just within the green boxes.} 
	%\heng{A smaller performance metric value (e.g., cpu usage) indicates a better performance.}} 
	}
	\label{fig:description}
\end{figure}


%Furthermore, prior studies do not consider the evolution aspect when configuration performance. 

Traditionally, prior work studied the difference in system performance caused by \emph{different} values of the \emph{same} option, without considering how the performance impact of an option evolves due to code changes~\cite{tse}. For instance, traditional approaches compare different values of a configuration option based on their raw performance values~\cite{RN2880,RN3537,RN3543}, as illustrated in Figure~\ref{fig:description-a}. However, such comparison is subjective as
the option's value \emph{V2} with worse performance might not necessarily be problematic, but might, as an example, just enable the execution of some extra features.

Conversely, even if an option's value has a good performance compared to other values, those differences in performance might start to vary % still be significantly different
when comparing to the performance of the same option value in the prior commit. Normally, one would expect to see the situation in Figure~\ref{fig:description-b}, which shows that both option values have a consistent variation in performance, in this case, a similar increase (regression) in the performance metric. In reality, one can observe %\bram{any indications for this, or at this point in the paper still a hypothesis?} \jinfu{hypothesis} 
cases such as in the example in Figure~\ref{fig:description-c}, where \emph{V1} still shows better performance compared to \emph{V2} after commit C2, but it faces a significantly larger performance regression compared to the prior commit than \emph{V2}. Similarly, in Figure~\ref{fig:description-d}, the \emph{V2} value still has a worse performance after the new commit, but its performance improved much more significantly compared to the prior commit than \emph{V1}.

%On the opposite side, an option's value that has a better performance compared to another value of the same option might be facing a large performance regression compared to the prior commit, as shown in the example of Figure~\ref{fig:description-d}. 
Therefore, different values of an option can have an inconsistent variation in terms of performance compared to the prior commit, which % . This happens when one or more values of an option exhibit a significant difference compared to the prior release. 
we refer to as \textbf{Inconsistent Option Performance Variation (a.k.a, \inconsistent)}. The \inconsistent might be problematic as it can hide a performance regression that is manifested when alerting configuration options. %In practice, all to often developers use the default value of a configuration option. However, the default value of a configuration option may cause performance regressions while other values of the same option indicates a performance improvement or no performance regression. %\bram{value? the ``hiding'' part is not clear}. 
%Similarly, when a default configuration shows a performance improvement, altering one of the options' value may indicate no performance improvement or even a performance regression.  
Such regressions can unfortunately go as unseen to the production environment. The \inconsistent may directly affect the user experience, increase the resources cost of the system and lead to reputational and financial repercussions. 
%\bram{so, the existing ``raw'' approaches would stop after seeing the improvement for the default value, or after seeing that V2 is still worse than V1, while they would ignore the change in variation => it is not clear here how bad such ignored variations are (the ``hiding'' is not clear)}

%In this paper, we instead normalize the performance by considering the prior commit for a tested version. In particular, we measure how much variance exists for an option value between a current version and its prior one. For example, an option X with a value Y in a new commit has a different performance consumption compared to the same option and value prior to the new commit. Then, we study the differences between these last variance to measure if different values for the same option manifest different performances compared to the prior change, which we refer to as ``Inconsistent Options Performance Variation'' (\inconsistent).

%In our prior work~\cite{DBLP:conf/icsm/ChenS17}, we conducted a first step toward assisting practitioners on identifying performance regression at the development stage. In particular, we repetitively execute the existing functional tests or micro-benchmarks for each commit. We identify performance regressions in each test or micro-benchmark if there exists statistically significant degradation with medium or large effect sizes in any performance metric.

In this paper, we perform a case study on two large-scale open-source software systems: \emph{Hadoop} and \emph{Cassandra}. We first conduct a preliminary study to quantify the prevalence of \inconsistent in practice. We observe that 81\% of the commits have at least one option manifesting an \inconsistent issue. We also observe that manually identifying such issues is challenging, as commits do not share the same options that manifest an \inconsistent. That motivates us to propose an automated model that predicts if the combination of a Commit, a Test, and an Option (\textbf{\instance}) would exhibit an \inconsistent issue. We evaluate our prediction model using the following two research questions: 

%Then, we propose a model to predict which Commit, Test, and Configuration (\textbf{\instance}) might suffer from an \inconsistent, which can be harmful. %the impact of software configuration on hiding performance regression and proposed a prediction model to identify such regressions at the development stage. In fact, software configuration makes the identification of performance regression more challenging. 
 %Therefore, we first quantify how often an \inconsistent can occur. Secondly, we propose and evaluate a prediction model that identifies whether a commit has a performance regression that is hidden under a certain configurations. %In particular, a source code change might not always show a regression, but just under certain configurations. Therefore, in this paper, we quantify the impact of software configurations on hiding performance regression. We then propose a prediction model that combines changed source code as well as configuration related metrics to predict whether a commit has a performance regression that might be hidden by a certain configurations. 
%We summarize our contribution in the following three research questions: 

\begin{itemize}
    
    \item \textbf{RQ1. \RQII}
    
    Our prediction model reaches an AUC up to 0.93 and 0.90 for predicting \inconsistent for \emph{Hadoop} and \emph{Cassandra}, respectively. We observe that random forest is the most performing model for four and three out of five performance measures (i.e., response time, CPU, memory, I/O Read, and I/O write) for \emph{Cassandra} and \emph{Hadoop}, respectively. 
    
    \item\textbf{RQ2. \RQIII}
    
    We observe that all four dimensions of metrics considered in our study, namely the code structure, code change, code token, and configuration options metrics, have a statistically significant impact in predicting \inconsistent. The dimensions that are related to the configuration options and the tokens of the changed code are the most important dimensions for both case studies. 
    
\end{itemize}

\noindent{Paper organization.} The rest of our paper is organized as follows: Section~\ref{sec:back} provides the background information and defines \inconsistent. Section~\ref{sec:datacollection} discusses our approach to conducting experiments and collecting data. Sections~\ref{sec:pq-results} and~\ref{sec:rq-results} present our results. Section~\ref{sec:threats} discusses the threats to the validity of our findings. Section~\ref{sec:relatedwork} provides prior work related to our paper. Finally, Section~\ref{sec:conclusion} concludes the paper.


%%% Local Variables:
%%% mode: latex
%%% TeX-master: "../main"
%%% End:

\input{tex/Background}
\input{tex/dataCollection}
\input{tex/preliminary}

\section{Predicting \inconsistent Problems} \label{sec:rq-results}

\subsection*{\textbf{RQ1. \RQII}}
\label{sec:rq2}

%\med{IMPORTANT NOTE: it's not ``caused'' by configuration, but just manifested under a subset of the existing configurations. The problem is not the configuration itself. }

%\med{we can add 2 RQs or enrich this RQ, by considering different dependent variables: if a config is different from the default config + when a config and a commit have a regression (which is the straightforward extension for Jinfu's prior work model) + what you already have, which is when we have a statistically significant variations of regression between different configurations}

%\med{todo: use option instead of parameter and make a clear distinction between configuration and option}
%\heng{Model is for per option, per commit, per test}\heng{TODO: update motivation and approach} 

%\heng{TODO: check and update results}
\noindent \textbf{Motivation.}
% \subsubsection*{Motivation}
The goal of this research question is to evaluate different classification approaches on predicting for which \instance one has to check multiple option's values. %whether a commit, test, and configuration (i.e., \instance) will spot a performance regression. 
In our preliminary study, we observe that the \inconsistent is common and hard to manually predict, %is not equally manifested throughout all the configurations. %For instance, even if the default configuration does not show any performance regression, other configurations can badly suffer from a performance regression. %different settings of the configuration options significantly impact the results of performance regression detection. 
which indicates that developers need to test different values for each option. However, as there are typically a large number of configuration options (e.g., \emph{Hadoop} version 2.7.3 has 355 configuration options) with different possible values, exhaustively experimenting with all different options for each test in performance testing is time- and resource-consuming. In this RQ, we aim to reduce the effort of conducting configuration-aware performance testing by predicting the need for testing with different values for a given configuration option when a code change is made (i.e., for a \instance). Specifically, our approach predicts whether a \instance manifests an \inconsistent, such that developers can make an informed decision on whether they should consider different values for that option in their performance testing.

\noindent \textbf{Approach.}
% \subsubsection*{Approach}
In this RQ, we build ML models to predict whether a \instance manifests an \inconsistent. % detection after a code change is committed by developers 
%(i.e., the need to adjusting the configuration value of an \instance). 
Below, we describe the detailed steps involved in our modeling process.

\noindent\textbf{Step 1. Data preparation.}

\textit{Step 1.1. Defining the target variable.} Our target variable is a binary variable that indicates whether a \instance manifests an \inconsistent, which we obtained following the approach discussed in Section~\ref{sec:datacollection}. In particular, after collecting performance measurements, we calculated the performance variation for each option, and discretize each \instance into \inconsistent or non-\inconsistent following the discretization approach of Section~\ref{sec:statisticalAnalysis}.
%For each \instance, we use the maximum difference between the performance regression detection results with different values of a configuration option as the target variable.
%First, for each \instance, %commit, each test, and each configuration option, 
%we run performance testing using all different values of the option (repeated 30 times) and record the maximum difference between the performance regression detection results (i.e., the Cliff's delta effect size) using the default option value and other option values.


\begin{table}[t]
     \caption{Overview of our selected metrics.}
    \label{tab:evaluatedSystems}
        \begin{tabular}{|p{1.2cm}|p{1.8cm}|p{4.5cm}|}
        \hline
        Dimension & Metric           & Rationale \\ %\hline
        \hline
\multirow{8}{*}{\begin{tabular}[c]{@{}l@{}}Code \\      change\end{tabular}}    & Number of modified subsystems & The more subsystems are changed, the higher risk the change may be~\cite{mockus2000predicting} \\ 
\cline{2-3} 
  & Number of modified directories & Changing   more directories may more likely introduce performance regressions~\cite{mockus2000predicting}.\\ \cline{2-3} 
  & Number of modified files & Changing many source files are more likely to cause performance regressions~\cite{Nagappan:2006:MMP}.\\
  \cline{2-3} 
  & Distribution of modified code across files & Scattered changes are more possible to introduce performance regressions~\cite{Hassan:2009:PFU}.\\ \cline{2-3} 
  & Number of modified methods & Changes   altering many methods are more likely to introduce performance regressions~\cite{Zimmermann:2007:PDE}. \\ \cline{2-3} 
  & Number of lines SOC in tests & Program   with more lines is more likely to suffer from performance regressions~\cite{Koru2009tse}.\\ \cline{2-3} 
  & Lines of code added & The   more lines of code added, the higher risk that the program will suffer from performance   regressions~\cite{Nagappan:2005:URC,Zimmermann:2007:PDE}.\\ 
  \cline{2-3} 
  & Lines of code deleted & The more lines of code deleted, the higher risk of performance regression is introduced~\cite{Nagappan:2005:URC,Zimmermann:2007:PDE}.\\ 
  \hline
\multirow{4}{*}{\begin{tabular}[c]{@{}l@{}}Code \\      structure\end{tabular}} & Number of methods in impacted test & Program with a large number of methods is more likely to suffer from performance regressions. \\ 
\cline{2-3} 
  & McCabe Cyclomatic complexity & Program with higher complexity is more likely to suffer from performance regressions~\cite{Hassan:2009:PFU}.\\ 
  \cline{2-3} 
  & Number of called subprograms & Large called subprograms will amplify the regression if there exist performance regressions in the called program~\cite{Nagappan:2006:MMP}.\\
  \cline{2-3} 
  & Number of calling subprograms & Large   calling subprograms will amplify the regressions if there exist performance regressions in the called program~\cite{Nagappan:2006:MMP}.\\ 
  \hline
Code token& Code tokens of the changed source code & Some code tokens may be more related to performance than other tokens.\\ 
\hline
Option token & Split configuration option names & The name components of a configuration option may be related to a specific performance metric.\\ \hline
\end{tabular}
\label{tab:metrics}
\end{table}
\begin{comment}

\begin{table}[t]
%\tabcolsep=0.1cm
\tiny
\caption{Overview of our selected metrics.%\heng{the code tokens also include tokens from the test file right?}
}
\begin{tabular}{|l|l|l|}
\hline
Dimension & Metric           & Rationale \\ \hline
\multirow{8}{*}{\begin{tabular}[c]{@{}l@{}}Code \\      change\end{tabular}}    & \begin{tabular}[c]{@{}l@{}}Number   of\\ modified\\      subsystems\end{tabular}   & \begin{tabular}[c]{@{}l@{}}The   more subsystems are changed, the higher risk\\      the change may be~\cite{mockus2000predicting}.\end{tabular}\\ \cline{2-3} 
  & \begin{tabular}[c]{@{}l@{}}Number of\\ modified\\      directories\end{tabular}    & \begin{tabular}[c]{@{}l@{}}Changing   more directories may more likely introduce\\      performance regressions~\cite{mockus2000predicting}.\end{tabular}     \\ \cline{2-3} 
  & \begin{tabular}[c]{@{}l@{}}Number of\\ modified\\      files\end{tabular}          & \begin{tabular}[c]{@{}l@{}}Changing   many source files are more likely to cause\\      performance regressions~\cite{Nagappan:2006:MMP}.\end{tabular}         \\ \cline{2-3} 
  & \begin{tabular}[c]{@{}l@{}}Distribution\\ of modified\\      code across files\end{tabular}      & \begin{tabular}[c]{@{}l@{}}Scattered   changes are more possible to introduce\\      performance regressions~\cite{Hassan:2009:PFU}.\end{tabular}\\ \cline{2-3} 
  & \begin{tabular}[c]{@{}l@{}}Number of\\ modified\\      methods\end{tabular}        & \begin{tabular}[c]{@{}l@{}}Changes   altering many methods are more likely \\      introduce performance regressions~\cite{Zimmermann:2007:PDE}.\end{tabular}  \\ \cline{2-3} 
  & \begin{tabular}[c]{@{}l@{}}Number of\\ lines SOC\\ in tests\end{tabular} & \begin{tabular}[c]{@{}l@{}}Program   with more lines is more likely to \\      suffer from performance regressions~\cite{Koru2009tse}.\end{tabular}            \\ \cline{2-3} 
  & \begin{tabular}[c]{@{}l@{}}Lines of\\ code added \end{tabular}      & \begin{tabular}[c]{@{}l@{}}The   more lines of code added, the higher risk that the\\      program will suffer from performance   regressions~\cite{Nagappan:2005:URC,Zimmermann:2007:PDE}.\end{tabular} \\ \cline{2-3} 
  & \begin{tabular}[c]{@{}l@{}}Lines of\\ code deleted \end{tabular} & \begin{tabular}[c]{@{}l@{}}The   more lines of code deleted, the higher risk of \\      performance regression is   introduced~\cite{Nagappan:2005:URC,Zimmermann:2007:PDE}.\end{tabular} \\ \hline
\multirow{4}{*}{\begin{tabular}[c]{@{}l@{}}Code \\      structure\end{tabular}} & \begin{tabular}[c]{@{}l@{}}Number of\\ methods in\\      impacted test\end{tabular}& \begin{tabular}[c]{@{}l@{}}Program with  a large number of methods is more\\      likely to suffer from performance regressions.\end{tabular}     \\ \cline{2-3} 
  & \begin{tabular}[c]{@{}l@{}}McCabe\\ Cyclomatic\\      complexity\end{tabular}      & \begin{tabular}[c]{@{}l@{}}Program   with higher complexity is more likely to suffer\\      from performance regressions~\cite{Hassan:2009:PFU}.\end{tabular}  \\ \cline{2-3} 
  & \begin{tabular}[c]{@{}l@{}}Number of\\ called\\      subprograms\end{tabular}      & \begin{tabular}[c]{@{}l@{}}Large   called subprograms will amplify the regression if \\      there exist performance regressions in the called   program~\cite{Nagappan:2006:MMP}.\end{tabular}         \\ \cline{2-3} 
  & \begin{tabular}[c]{@{}l@{}}Number of\\ calling\\      subprograms\end{tabular}     & \begin{tabular}[c]{@{}l@{}}Large   calling subprograms will amplify the regressions if \\      there exist performance regressions in the called   program~\cite{Nagappan:2006:MMP}.\end{tabular}        \\ \hline
Code token& \begin{tabular}[c]{@{}l@{}}Code   tokens\\ of the changed\\ source code\end{tabular}        & \begin{tabular}[c]{@{}l@{}}Some   code tokens may be more related to performance \\      than other tokens.\end{tabular}          \\ \hline
\begin{tabular}[c]{@{}l@{}}Configuration\\       option\end{tabular}            & \begin{tabular}[c]{@{}l@{}}Split   configuration\\      option names\end{tabular}            & \begin{tabular}[c]{@{}l@{}}The name components of a configuration option may be \\      related to a specific performance metric.\end{tabular}             \\ \hline
\end{tabular}
\label{tab:metrics}
\end{table}

\end{comment}

\textit{Step 1.2. Selecting the features.}
We consider four dimensions of software metrics that are related to the likelihood of a configuration option impacting the performance testing of a code commit for each test (i.e., of a \instance). Table~\ref{tab:metrics} lists the detailed metrics used in our models. %\med{the hypothesis that link the four dimensions to the configuration performance regression doesn't make sense. I would suggest to say that ``
We already found in our prior work~\cite{jinfu_tse2020} that code structure, and code change dimensions are important for predicting performance regressions, but there we did not consider the impact of different configurations on the manifestation of performance regressions. Therefore, we use the prior dimensions as well as an additional dimension about the configuration options.

%\med{To reomve the following itemize about the dimensions:}
%\begin{itemize}
    %\item \heng{todo later: motivate the choice of metrics from RQ1 findings and Jinfu's prior performance regression model}
    \textbf{Code change metrics}. This dimension contains metrics that capture the code changes in a commit (e.g., the number of changed lines of code). Some code changes (e.g., big code changes) might be more likely to impact how different values of a configuration option make a difference in performance testing.
    
    \textbf{Code structure metrics}. This dimension contains metrics that describe the static structure of the source code files (e.g., cyclomatic complexity). %\med{not clear why? how a code change is related to a configuration? \jinfu{We have option-test mapping.}}
    Our intuition is that changing certain code (e.g., complex code) is more likely to alter the performance impact of a configuration option.
    
    \textbf{Code token metrics}. This dimension considers the code tokens of the methods that are changed in the commit and the test file. %\heng{can we separate the tokens from the source code and the test files into two dimensions?}. 
    Some code tokens (e.g., ``speed'') may be more related to performance than other tokens. We use the \emph{lscp}\footnote{Lscp is a lightweight source code preprocessor that can be used to isolate and manipulate the linguistic data (i.e., identifier names, comments, and string literals) from the source code: \url{https://github.com/doofuslarge/lscp}} tool to extract the code tokens from the source code. 
    
    \textbf{Configuration option metrics}. This dimension considers the characteristics of the configuration option, in particular the tokens in the configuration option names. We assume that some options (e.g., system resource-related options) are more likely to impact the performance regressions detection results of a commit.
%\end{itemize}


\textit{Step 1.3. Pre-processing the features.}
The code token metrics include thousands of unique code tokens. Thus, we need to pre-process such metrics into a numeric representation. We consider three different approaches to pre-process the code token metrics. 
\begin{itemize}
    \item Term frequency-inverse document frequency (tf-idf). Tf-idf~\cite{ramos2003using} generates a feature for each unique token. The value of a feature for a commit is the term frequency of the corresponding token (i.e., $tf(t,c) = f_{t,c}$, where $f_{t,c}$ is the number of times a token $t$ appears in commit $c$) times the inverse frequency of the commits that contain the token ($idf(t) = \log{(N/N_t)}$, where $N$ is the total number of commits while $N_t$ is the number of commits containing the token $t$.) %heng{to confirm the detailed calculation}
    
    \item Principal component analysis (PCA). Using tf-idf generates a large %\bram{than what?} \jinfu{traditional features}
    number of features that may lead to very complex models. Therefore, we apply PCA~\cite{wold1987principal} on the features resulting from tf-idf to reduce the number of features. Specifically, we only consider the top principal components that contribute to 95\% of the variance in the features together.
    
    \item Word embeddings. We use word2vec~\cite{Mikolov:2013:DRW:2999792.2999959,Mikolov2013} to code each token into a vector of 128 numerical values. Specifically, we pre-train the embeddings from a large code base\footnote{\url{https://doi.org/10.5281/zenodo.3801975}}, %\bram{is that still anonymous, or are authors known by now?}, 
    then apply the pre-trained embeddings on the tokens in our data. Then we use a mean aggregation of the vectors representing the tokens in a given commit.
    
    
    
\end{itemize}

\noindent\textbf{Step 2. Model construction.}
%\textit{Considered ML algorithms.}
%\textit{Model training.}
We build %\jinfu{three types of models, including logistic regression, random forest, and XGBosst}, 
machine learning models to predict whether a configuration option suffers from an \inconsistent on a given \instance. %a code commit using a test case. 
For the generalization of our results, we consider five different types of models, including random forest (RF), logistic regression (LR), XGBoost (XG), neural network (NN), and convolutional neural network (CNN). %, which are widely used in software engineering studies~\heng{cite}. 
A random forest is a classifier consisting of a collection of decision tree classifiers and each tree casts a vote for the most popular class for a given input~\cite{breiman2001random}. 
%Each internal tree is constructed using a different bootstrap sample of the input data. Random forests are naturally robust against overfitting~\cite{breiman2001random} and they usually perform very well for software engineering tasks~\cite{ghotra2015revisiting, DBLP:journals/tse/Tantithamthavorn17,DBLP:journals/tosem/LiJLHHHZWC20,DBLP:journals/ese/LiSZH17}. 
%\med{we don't have to say this. somebody might argue why trying all these models when ~\cite{ghotra2015revisiting} already found that random forest is the best}Prior work~\cite{ghotra2015revisiting} compares 31 classifiers in software defects prediction and suggests that random forests outperform other classifiers. %Besides, random forests provide us a way to do sensitivity analysis on the measures so that we can understand the most influential factors in our models~\cite{breiman2002manual,liaw2002classification}.
Logistic regression is a statistical model that uses a logit function to model a binary variable (the target variable) as a linear combination of the independent variables~\cite{hosmer2013applied}, which is widely used in software analytics~\cite{tantithamthavorn2018impact,shang2015automated}. 
XGBoost is an efficient and accurate implementation of the gradient boosting algorithm~\cite{chen2016xgboost}, which is reported to perform better than other machine learning models in software engineering applications~\cite{Liao2020LogPerfReg}.
The neural network model~\cite{DBLP:journals/jmlr/GlorotBB11} used in our study consists of four layers and is trained with batch size 100, and 10 epochs.
The CNN model~\cite{DBLP:journals/tnn/LawrenceGTB97} in our study consists of five layers, and are trained with batch size 100, and 10 epochs.


Prior to constructing our models, we check the pairwise correlation between our features using the Pearson correlation test (\(\rho\))~\cite{benesty2009pearson}. We choose the Pearson correlation method because it is robust to non-normally distributed data. In this work, we choose the correlation value $0.7$ as the threshold to remove collinearity. In other words, if the correlation between a pair of features is greater than 0.7 (\(|\rho|>0.7\)), we only keep one of the two features in the model.
%\med{what about redundancy analysis?} 
We then perform a redundancy analysis on the features. In particular, we use each feature as a dependent variable and use the remaining features as independent variables to build a regression model and calculate the $R^2$ of each model. If the $R^2$ is more than 0.9~\cite{markASE}, the current dependent variable is considered redundant. 

\noindent\textbf{Step 3. Model evaluation.}
%\textit{Evaluation metrics.}
%\textit{Ten-fold cross-validation.}
We use 10-fold cross-validation to evaluate the performance of our models. %\bram{did we evaluate actual prediction? if not, we should talk about ``explanatory models'' instead of ``prediction models''}
In each repetition of the 10-fold cross-validation, the whole data set is randomly partitioned into 10 sets of roughly equal size. One subset is used as the testing set (i.e., the held-out set) and the other nine subsets are used as the training set. 
We train our models using the training set and evaluate the performance of our models on the held-out set.
The process repeats 10 times until all subsets are used as a testing set once.

In each fold of the cross-validation, we use precision, recall and 
AUC to measure the performance of our models.
Precision measures the ratio of cases when a configuration option actually impacts the performance regression detection among all the cases that our models predict to adjust a configuration option (i.e., $\frac{\textnormal{true positives}}{\textnormal{true positives} + \textnormal{false positives}}$). Recall measures the ratio of cases when our models predict to adjust a configuration option among all the cases when a configuration option actually impacts the performance regressions detection (i.e., $\frac{\textnormal{true positives}}{\textnormal{true positives} + \textnormal{false negatives}}$). 
AUC measures our models' ability to discriminate the \instance cases into \inconsistent and non-\inconsistent cases. Specifically, AUC is the area under the ROC curve, which plots the true positive rate against the false positive rate under different classification thresholds. 
Prior work recommends the use of AUC over threshold-dependent measures (e.g., precision and recall) when measuring the model performance~\cite{tantithamthavorn18experience}.
%\med{we should instead say data is not equally distributed, so precision and recall are not good performance metrics + drop precision and recall from the tables.}We also included the evaluation results of other metrics (precision and recall) in our replication package~\heng{to do: add replication package}. \jinfu{After update Precision and recall in NN and CNN, we can keep precision and recall.}

\noindent \textbf{Result.}
% \subsubsection*{Results} 
%\heng{Check RQ2 results}
%\med{what about adding (1) a model that predicts when a \instance suffers from an \inconsistent on at least one performance measure and (3) another model on all the performance measures. the goal of this is to simulate different use cases. the first use case (1) is for people who are interested on any performance metric, what (2) we already have is for people who are interested on one precise performance metric, and the (3) is for people who are looking for very bad options that cause problems on all the performance metrics. } \jinfu{If practitioners want to predict whether a \instance suffers from \inconsistent in any of performance metric, practitioners can combine the five predictors. In other words, any of the model in five performance metrics predicts a \instance suffers from \inconsistent.}
\noindent \textbf{Our models can effectively predict when a \instance is manifesting an \inconsistent for all of our five studied performance measures} (as shown in Table 1 %~\ref{tab:model_evaluation_hadoop} 
and Table 2 %~\ref{tab:model_evaluation_cassandra}
in the appendix). % show the performance of our models for predicting whether adjusting a configuration option would cause difference in the results of performance regression detection. 
Our best models (i.e., as indicated by the \textbf{\textit{bold-italic}} values) achieve an AUC of 0.85 to 0.94 on the \emph{Hadoop} project and 0.79 to 0.90 on the \emph{Cassandra} project, for different performance metrics.
For the \emph{Hadoop} project, %~\heng{make the wording system or project consistent throughout the paper}, 
RF is the best model for four out of the five performance metrics, achieving an AUC of 0.85 to 0.93. Even if XG shows the best AUC performance for the fifth performance metric (i.e., Response time), the difference between RF and XG is only 0.01. 
For the \emph{Cassandra} project, RF shows the best performance on three out of five performance metrics. NN shows the best performance on also three performance metrics (Memory and I/O read have the same performance as the RF model). The average AUC of the best NN model is 0.83, while the average AUC of the best RF model is 0.82. 
Note that %\med{is the following correct?} 
NN, on the other side, requires a large amount of resources to train and test a model, while the improvements it shows over RF is trivial.
CNN shows the best performance on only one performance metric (i.e., with an AUC of 0.79 for the Response time). However, the average AUC of the best CNN model is 0.09 lower than that of RF. 
%Note that NN and RF show similar AUC performance on two performance metrics. % the best models for different performance metrics. 
%In summary, most of the performance metrics can be effectively predicted by RF, while other metrics show a low AUC performance differences with other models. 
In summary, we suggest that developers consider the RF model for predicting when a \instance has an \inconsistent problem. 
%, a precision of 0.72 to 0.79, a recall of 0.33 to 0.60 for the Hadoop project, and an AUC of 0.75 to 0.86, a precision of 0.67 to 0.77, a recall of 0.46 to 0.65 for the Cassandra project.
%an average \emph{AUC} of 0.83 and 0.75 in \emph{Hadoop} and \emph{Cassandra}, respectively. 

%\med{I don't see any value on this comparison. both of them are good}\noindent\textbf{Our models are better at distinguishing the \instance cases that need adjustments of the configuration options and that do not for the Hadoop project than for the Cassandra project.} 
%Table~\ref{tab:model_evaluation_hadoop} and Table~\ref{tab:model_evaluation_cassandra} show that 
Our best models achieve a better performance for the \emph{Hadoop} system 
%(with an AUC of 0.85 to 0.94) 
than for the \emph{Cassandra} system. % (with an AUC of 0.79 to 0.90).  %for the Hadoop project (best average AUC is 0.91) and a lower AUC of 0.84 to 0.90 for the Cassandra project (best average AUC is 0.83). 
As discussed in RQ1 of our preliminary study, the different commits show more inconsistent $<$test, option, \inconsistent$>$ triplets (i.e., more dark cells in Figure~\ref{fig:across-commit-cassandra}) in the \emph{Cassandra} system than in the \emph{Hadoop} system, i.e., most of the commits different from each other. Therefore, it is more difficult to predict \inconsistent for the \emph{Cassandra} system, which could explain the reason that our models perform better for the \emph{Hadoop} project. %\bram{in that figure for Cassandra, is every commit totally different from each other?} 
%In particular, for the Hadoop project, our models achieve the best performance for predicting the impact of adjusting configurations on the detection of response time regressions (with best AUC of 0.97) and worst performance on I/O write regressions (with best AUC of 0.85). For the Cassandra project, our models achieve more consistent performance across different performance metrics.

\noindent \textbf{The choice of representation of the code tokens significantly impacts the performance of our models.} For the traditional models (RF, LR, and XG), using code embeddings to represent the code tokens often achieves the best performance, while using PCA usually results in the worst performance (as shown in Table 1 and Table 2 in the appendix). For example, for the \emph{Hadoop} project, the RF model achieves an AUC of 0.85 to 0.93 using code embeddings, 0.82 to 0.93 using tf-idf, and only 0.59 to 0.76 using PCA. The reason for the poor performance of the models using PCA might be that PCA significantly reduced the information in the tokens through dimension reduction, even though we considered the principal components that account for 95\% of the variance in the original variables.

In contrast, for the deep neural network models (NN and CNN), using PCA to represent the code tokens may achieve better results than the other two representations. For example, for the \emph{Cassandra} project, the CNN model combined with PCA achieves the best AUC for two out of the five performance metrics, across all different models.
The reason might be that there are much more options in our studied systems %\bram{in what sense?} 
in the deep neural network models, while using PCA could significantly reduce the number of options to be trained.

%\noindent \textbf{Our models can be used to equally predict performance regression introduced by adjusting configuration option in all performance metrics.} By examining the prediction results with all performance metrics, we find that all the performance metrics have similar \emph{AUC}s. 
%\noindent \textbf{For the Hadoop project, our models achieve the best performance for predicting the impact of adjusting configurations on the detection of response time regressions and worst performance on I/O write regressions. For the Cassandra project, our models achieve more consistent performance across different performance metrics.} As shown in Table~\ref{tab:model_evaluation_hadoop} and Table~\ref{tab:model_evaluation_cassandra}, for the Hadoop project, our best models achieve an AUC of 0.97 for the response time metric and an AUC of 0.85 for the I/O write metric. 
%In comparison, our best models achieve an AUC of 0.84 to 0.90 for the different performance metrics of the Cassandra project.
%our RF model with code embeddings achieves an AUC of 0.93 on reponse time regressions and an AUC of 0.85 on I/O write regressions. On the contrary, the same model achieves an AUC of 0.74 on response time regressions and an AUC of 0.85 on I/O write regressions.
%The difference can be resulted from the different characteristics of the two systems: as a distributed computing platform, the response time of Hadoop is sensitive to many factors including the configuration options; as a database system, there are frequent I/O write in the Cassandra system, thus the I/O write performance could be significantly impacted by the configuration options.

%\heng{Using tree-based models (RF and XGBoost together with TF-IDF or code embeddings for encoding the code tokens achieves the best performance for predicting whether a configuration will have a significant impact on the performance testing results.}

%\heng{Add a discussion referring to the heatmaps in Figure 3 and Figure 5 to discuss why Hadoop gets better performance than Cassandra: the impact of the configuration options are more consistent in Hadoop.}


\vspace{0.5cm}
\begin{Summary}{Summary of RQ1}{}
Our models can effectively predict whether a \instance manifests an \inconsistent problem. Random forest based on code embedding shows the best performance on predicting \inconsistent for most of the performance measures.
\end{Summary}


%%% Local Variables:
%%% mode: latex
%%% TeX-master: "../main"
%%% End:


\section{Predicting \inconsistent Problems} \label{sec:rq-results}

\subsection*{\textbf{RQ1. \RQII}}
\label{sec:rq2}

%\med{IMPORTANT NOTE: it's not ``caused'' by configuration, but just manifested under a subset of the existing configurations. The problem is not the configuration itself. }

%\med{we can add 2 RQs or enrich this RQ, by considering different dependent variables: if a config is different from the default config + when a config and a commit have a regression (which is the straightforward extension for Jinfu's prior work model) + what you already have, which is when we have a statistically significant variations of regression between different configurations}

%\med{todo: use option instead of parameter and make a clear distinction between configuration and option}
%\heng{Model is for per option, per commit, per test}\heng{TODO: update motivation and approach} 

%\heng{TODO: check and update results}

\subsubsection*{Motivation}

The goal of this research question is to evaluate different classification approaches on predicting for which \instance one has to check multiple option's values. %whether a commit, test, and configuration (i.e., \instance) will spot a performance regression. 
In our preliminary study, we observe that the \inconsistent is common and hard to manually predict, %is not equally manifested throughout all the configurations. %For instance, even if the default configuration does not show any performance regression, other configurations can badly suffer from a performance regression. %different settings of the configuration options significantly impact the results of performance regression detection. 
which indicates that developers need to test different values for each option. However, as there are typically a large number of configuration options (e.g., \emph{Hadoop} version 2.7.3 has 355 configuration options) with different possible values, exhaustively experimenting with all different options for each test in performance testing is time and resource-consuming. In this RQ, we aim to reduce the effort of conducting configuration-aware performance testing by predicting the need for adjusting a configuration option for a test when a code change is made (i.e., for a \instance). Specifically, our approach predicts whether a \instance manifests an \inconsistent, such that developers can make an informed decision on whether they should consider different values for that  option in their performance testing.

\subsubsection*{Approach}

In this RQ, we build ML models to predict whether a \instance manifests an \inconsistent. % detection after a code change is committed by developers 
%(i.e., the need to adjusting the configuration value of an \instance). 
Below, we describe the detailed steps involved in our modeling process.

\noindent\textbf{Step 1. Data preparation.}

\textit{Step 1.1. Defining the target variable.} Our target variable is a binary variable that indicates whether an \instance manifests an \inconsistent, which we obtained following the approach discussed in Section~\ref{sec:datacollection}. In particular, after collecting performance measurements, we calculated the performance variation for each option, and discretize each \instance into a \inconsistent or a non \inconsistent following the discretization approach of Section~\ref{sec:statisticalAnalysis}.
%For each \instance, we use the maximum difference between the performance regression detection results with different values of a configuration option as the target variable.
%First, for each \instance, %commit, each test, and each configuration option, 
%we run performance testing using all different values of the option (repeated 30 times) and record the maximum difference between the performance regression detection results (i.e., the Cliff's delta effect size) using the default option value and other option values.


\begin{table}[htbp]
\tabcolsep=0.05cm
\scriptsize
\caption{Overview of our selected metrics. %\heng{the code tokens also include tokens from the test file right?}
}
\begin{tabular}{|l|l|l|}
\hline
Dimension & Metric           & Rationale \\ \hline
\multirow{8}{*}{\begin{tabular}[c]{@{}l@{}}Code \\      change\end{tabular}}    & \begin{tabular}[c]{@{}l@{}}Number   of modified\\      subsystems\end{tabular}   & \begin{tabular}[c]{@{}l@{}}The   more subsystems are changed, the higher risk\\      the change may be~\cite{mockus2000predicting}.\end{tabular}\\ \cline{2-3} 
  & \begin{tabular}[c]{@{}l@{}}Number of modified\\      directories\end{tabular}    & \begin{tabular}[c]{@{}l@{}}Changing   more directories may more likely introduce\\      performance regressions~\cite{mockus2000predicting}.\end{tabular}     \\ \cline{2-3} 
  & \begin{tabular}[c]{@{}l@{}}Number of modified\\      files\end{tabular}          & \begin{tabular}[c]{@{}l@{}}Changing   many source files are more likely to cause\\      performance regressions~\cite{Nagappan:2006:MMP}.\end{tabular}         \\ \cline{2-3} 
  & \begin{tabular}[c]{@{}l@{}}Distribution of modified\\      code across files\end{tabular}      & \begin{tabular}[c]{@{}l@{}}Scattered   changes are more possible to introduce\\      performance regressions~\cite{Hassan:2009:PFU}.\end{tabular}\\ \cline{2-3} 
  & \begin{tabular}[c]{@{}l@{}}Number of modified\\      methods\end{tabular}        & \begin{tabular}[c]{@{}l@{}}Changes   altering many methods are more likely \\      introduce performance regressions~\cite{Zimmermann:2007:PDE}.\end{tabular}  \\ \cline{2-3} 
  & \begin{tabular}[c]{@{}l@{}}Number of lines SOC\\ in tests\end{tabular} & \begin{tabular}[c]{@{}l@{}}Program   with more lines is more likely to \\      suffer from performance regressions~\cite{Koru2009tse}.\end{tabular}            \\ \cline{2-3} 
  & Lines of code added      & \begin{tabular}[c]{@{}l@{}}The   more lines of code added, the higher risk that the\\      program will suffer from performance   regressions~\cite{Nagappan:2005:URC,Zimmermann:2007:PDE}.\end{tabular} \\ \cline{2-3} 
  & Lines of code deleted    & \begin{tabular}[c]{@{}l@{}}The   more lines of code deleted, the higher risk of \\      performance regression is   introduced~\cite{Nagappan:2005:URC,Zimmermann:2007:PDE}.\end{tabular} \\ \hline
\multirow{4}{*}{\begin{tabular}[c]{@{}l@{}}Code \\      structure\end{tabular}} & \begin{tabular}[c]{@{}l@{}}Number of methods in\\      impacted test\end{tabular}& \begin{tabular}[c]{@{}l@{}}Program with   large number of methods is more\\      likely to suffer from performance regressions.\end{tabular}     \\ \cline{2-3} 
  & \begin{tabular}[c]{@{}l@{}}McCabe Cyclomatic\\      complexity\end{tabular}      & \begin{tabular}[c]{@{}l@{}}Program   with higher complexity is more likely to suffer\\      from performance regressions~\cite{Hassan:2009:PFU}.\end{tabular}  \\ \cline{2-3} 
  & \begin{tabular}[c]{@{}l@{}}Number of called\\      subprograms\end{tabular}      & \begin{tabular}[c]{@{}l@{}}Large   called subprograms will amplify the regression if \\      there exists performance regressions in the called   program~\cite{Nagappan:2006:MMP}.\end{tabular}         \\ \cline{2-3} 
  & \begin{tabular}[c]{@{}l@{}}Number of calling\\      subprograms\end{tabular}     & \begin{tabular}[c]{@{}l@{}}Large   calling subprograms will amplify the regressions if \\      there exists performance regressions in the called   program~\cite{Nagappan:2006:MMP}.\end{tabular}        \\ \hline
Code token& \begin{tabular}[c]{@{}l@{}}Code   tokens of the\\      changed source code\end{tabular}        & \begin{tabular}[c]{@{}l@{}}Some   code tokens may be more related to performance \\      than other tokens.\end{tabular}          \\ \hline
\begin{tabular}[c]{@{}l@{}}Configuration\\       option\end{tabular}            & \begin{tabular}[c]{@{}l@{}}Splited   configuration\\      option names\end{tabular}            & \begin{tabular}[c]{@{}l@{}}The   name of configuration option may be related to a \\      specific performance metric.\end{tabular}             \\ \hline
\end{tabular}
\label{tab:metrics}
\end{table}

\textit{Step 1.2. Selecting the features.}
We consider four dimensions of software metrics that are related to the likelihood of a configuration option impacting the performance testing of a code commit for each test (i.e., of an \instance). Table~\ref{tab:metrics} lists the detailed metrics used in our models. %\med{the hypothesis that link the four dimensions to the configuration performance regression doesn't make sense. I would suggest to say that ``
We already found in our prior work~\cite{jinfu_tse2020} that code structure, and code change dimensions are important for predicting performance regressions, but we did not consider the impact of different configurations on the manifestation of performance regressions. Therefore, we use the prior dimensions as well as an additional dimension about the configuration options.

%\med{To reomve the following itemize about the dimensions:}
%\begin{itemize}
    %\item \heng{todo later: motivate the choice of metrics from RQ1 findings and Jinfu's prior performance regression model}
    \textbf{Code change metrics}. This dimension contains metrics that capture the code changes in a commit (e.g., the number of changed lines of code). Some code changes (e.g., big code changes) might be more likely to impact how different values of a configuration option make difference in performance testing.
    
    \textbf{Code structure metrics}. This dimension contains metrics that describe the static structure of the source code files (e.g., Cyclomatic complexity). %\med{not clear why? how a code change is related to a configuration? \jinfu{We have option-test mapping.}}
    Our intuition is that changing certain code (e.g., complex code) is more likely to alter the performance impact of a configuration option.
    
    \textbf{Code token metrics}. This dimension considers the code tokens of the methods that are changed in the commit and the test file.%\heng{can we separate the tokens from the source code and the test files into two dimensions?}. 
    Some code tokens (e.g., ``speed'') may be more related the performance than other tokens. We use the \emph{lscp}\footnote{Lscp is a lightweight source code preprocesser that can be used to isolate and manipulate the linguistic data (i.e., identifier names, comments, and string literals) from the source code: \url{https://github.com/doofuslarge/lscp}} tool to extract the code tokens from the source code. 
    
    \textbf{Configuration option metrics}. This dimension considers the characteristics of the configuration option (e.g., the tokens in the configuration option). We assume that some options (e.g., system resource-related options) are more likely to impact the performance regressions detection results of a commit.
%\end{itemize}


\textit{Step 1.3. Pre-processing the features.}
The code token metrics include thousands of unique code tokens. Thus, we need to pre-process such metrics into a numeric representation. We consider three approaches to pre-process the code token metrics. 
\begin{itemize}
    \item Term frequency–inverse document frequency (tf-idf). Tf-idf~\cite{ramos2003using} generates a feature for each unique token. The value of a feature for a commit is the term frequency of the corresponding token (i.e., $tf(t,c) = f_{t,c}$, where $f_{t,c}$ is the number of times a token $t$ appears in commit $c$) times the inverse frequency of the commits that contain the token ($idf(t) = \log{(N/N_t)}$, where $N$ is the total number of commits while $N_t$ is the number of commits containing the token $t$.) %heng{to confirm the detailed calculation}
    
    \item Principal component analysis (PCA). Using tf-idf generates a larger number of features that may lead to very complex models. Therefore, we apply PCA~\cite{wold1987principal} on the features resulted from tf-idf to reduce the number of features. Specifically, we only consider the top principal components that contribute to 95\% of the variance in the features together.
    
    \item Word embeddings. We use word2vec~\cite{Mikolov:2013:DRW:2999792.2999959,Mikolov2013} to code each token into a vector of 128 numerical values. Specifically, we pre-train the embeddings from a large code base\footnote{\url{https://doi.org/10.5281/zenodo.3801975}} then applying the pre-trained embeddings on the tokens in our data. Then we use a mean aggregation of the vectors representing the tokens in a commit.
    
    
    
\end{itemize}

\noindent\textbf{Step 2. Model construction.}
%\textit{Considered ML algorithms.}
%\textit{Model training.}
We build %\jinfu{three types of models, including logistic regression, random forest, and XGBosst}, 
machine learning models to predict whether a configuration option suffers from an \inconsistent on a given \instance. %a code commit using a test case. 
For the generalization of our results, we consider five different types of models, including random forest (RF), logistic regression (LR), XGBoost (XG), neural network (NN), and convolutional neural network (CNN). %, which are widely used in software engineering studies~\heng{cite}. 
A random forest is a classifier consisting of a collection of decision tree classifiers and each tree casts a vote for the most popular class for a given input~\cite{breiman2001random}. 
%Each internal tree is constructed using a different bootstrap sample of the input data. Random forests are naturally robust against overfitting~\cite{breiman2001random} and they usually perform very well for software engineering tasks~\cite{ghotra2015revisiting, DBLP:journals/tse/Tantithamthavorn17,DBLP:journals/tosem/LiJLHHHZWC20,DBLP:journals/ese/LiSZH17}. 
%\med{we don't have to say this. somebody might argue why trying all these models when ~\cite{ghotra2015revisiting} already found that random forest is the best}Prior work~\cite{ghotra2015revisiting} compares 31 classifiers in software defects prediction and suggests that random forests outperform other classifiers. %Besides, random forests provide us a way to do sensitivity analysis on the measures so that we can understand the most influential factors in our models~\cite{breiman2002manual,liaw2002classification}.
A logistic regression is a statistical model that uses a logit function to model a binary variable (the target variable) as a linear combination of the independent variables~\cite{hosmer2013applied}, which is widely used in software analytics~\cite{tantithamthavorn2018impact,shang2015automated}. 
XGBoost is an efficient and accurate implementation of the gradient boosting algorithm~\cite{chen2016xgboost}, which is reported to perform better than other machine learning models in software engineering applications~\cite{Liao2020LogPerfReg}.
The neural network model~\cite{DBLP:journals/jmlr/GlorotBB11} used in our study consists of four layers and are trained with 100 batch size, and 10 epochs.
The CNN model~\cite{DBLP:journals/tnn/LawrenceGTB97} in our study consists of five layers, and are trained with 100 batch size, and 10 epochs.


Prior to constructing our models, we check the pairwise correlation between our features using the Pearson correlation test (\(\rho\))~\cite{benesty2009pearson}. We choose the Pearson correlation method because it is robust to non-normally distributed data. In this work, we choose the correlation value $0.7$ as the threshold to remove collinearity. In other words, if the correlation between a pair of features is greater than 0.7 (\(|\rho|>0.7\)), we keep one of the two features in the model.
%\med{what about redundancy analysis?} 
We then perform redundancy analysis on the features. In particular, we use each feature as a dependent variable and use the remaining features as independent variables to build a regression model and calculate the $R^2$ of each model. If the $R^2$ is more than 0.9~\cite{markASE}, the current dependent variable is considered redundant. 

\noindent\textbf{Step 3. Model evaluation.}
%\textit{Evaluation metrics.}
%\textit{Ten-fold cross-validation.}
We use a 10-fold cross-validation to evaluate the performance of our models.
In each repetition of the 10-fold cross-validation, the whole data set is randomly partitioned into 10 sets of roughly equal size. One subset is used as the testing set (i.e., the held-out set) and the other nine subsets are used as the training set. 
We train our models using the training set and evaluate the performance of our models on the held-out set.
The process repeats 10 times until all subsets are used as a testing set once.

In each fold of the cross-validation, we use precision, recall, and 
AUC to measure the performance of our models.
Precision measures the ratio of cases when a configuration option actually impacts the performance regressions detection among all the cases that our models predict to adjust a configuration option (i.e., $\frac{\textnormal{true positives}}{\textnormal{true positives} + \textnormal{false positives}}$). Recall measures the ratio of cases when our models predict to adjust a configuration option among all the cases when a configuration option actually impacts the performance regressions detection (i.e., $\frac{\textnormal{true positives}}{\textnormal{true positives} + \textnormal{false negatives}}$). 
AUC measures our models' ability to discriminate the \instance cases into \inconsistent and non \inconsistent cases. Specifically, AUC is the area under the ROC curve which plots the true positive rate against the false positive rate under different classification thresholds. 
Prior work recommends the use of AUC over threshold-dependent measures (e.g., precision and recall) when measuring the model performance~\cite{tantithamthavorn18experience}.
%\med{we should instead say data is not equally distributed, so precision and recall are not good performance metrics + drop precision and recall from the tables.}We also included the evaluation results of other metrics (precision and recall) in our replication package~\heng{to do: add replication package}. \jinfu{After update Precision and recall in NN and CNN, we can keep precision and recall.}


\subsubsection*{Results} 
%\heng{Check RQ2 results}

%\heng{update the numbers}
%\heng{We only keep AUC in the paper and put all metrics in the replication package.}
\begin{table}
 \tabcolsep=0.18cm
\caption{\emph{Hadoop}'s results of using different models to predict whether configuration options cause the manifesting of performance regressions. The best results for each performance metric and each model are highlighted in \textit{italic}. The best results for each performance metric across different models are highlighted in \textbf{\textit{bold-italic}}.} %\heng{Still a few results show very low precision and recall but high AUC (e.g., NN with tf-idf on Hadoop for Res. time and CPU, CNN with tf-idf on Hadoop for CPU} \heng{Todo: only keep AUC} \jinfu{Update Table6 and Table7}}

\begin{tabular}{|c|r|r|r|r|r|r|r|r|r|}
\hline
\multicolumn{10}{|c|}{Hadoop}    \\ \hline
\multirow{2}{*}{} & \multicolumn{3}{c|}{RF with tf-idf}        & \multicolumn{3}{c|}{RF with PCA}           & \multicolumn{3}{c|}{RF with code embedding}\\ \cline{2-10} 
                  & \multicolumn{1}{c|}{Pre.} & \multicolumn{1}{c|}{Recall} & \multicolumn{1}{c|}{AUC} & \multicolumn{1}{c|}{Pre.} & \multicolumn{1}{c|}{Recall} & \multicolumn{1}{c|}{AUC} & \multicolumn{1}{c|}{Pre.} & \multicolumn{1}{c|}{Recall} & \multicolumn{1}{c|}{AUC} \\ \hline
Res. time         & 0.68  & 0.39    & \textit{0.93}            & 0.68  & 0.39    & 0.66 & 0.73  & 0.33    & \textit{0.93}            \\ \hline
Cpu               & 0.70  & 0.51    & 0.90 & 0.55  & 0.02    & 0.71 & 0.77  & 0.60    & \textit{\textbf{0.92}}   \\ \hline
Memory            & 0.64  & 0.36    & 0.87 & 0.48  & 0.04    & 0.69 & 0.75  & 0.41    & \textit{\textbf{0.91}}   \\ \hline
I/O Read          & 0.68  & 0.54    & 0.91 & 0.58  & 0.02    & 0.76 & 0.79  & 0.56    & \textit{\textbf{0.93}}   \\ \hline
I/O Write         & 0.63  & 0.44    & 0.82 & 0.44  & 0.02    & 0.59 & 0.72  & 0.49    & \textit{\textbf{0.85}}   \\ \hline
Average           & 0.67  & 0.45    & 0.89 & 0.55  & 0.10    & 0.68 & 0.75  & 0.48    & \textit{\textbf{0.91}}   \\ \hline
\multirow{2}{*}{} & \multicolumn{3}{c|}{LR with tf-idf}        & \multicolumn{3}{c|}{LR with PCA}           & \multicolumn{3}{c|}{LR with code embedding}\\ \cline{2-10} 
                  & \multicolumn{1}{c|}{Pre.} & \multicolumn{1}{c|}{Recall} & \multicolumn{1}{c|}{AUC} & \multicolumn{1}{c|}{Pre.} & \multicolumn{1}{c|}{Recall} & \multicolumn{1}{c|}{AUC} & \multicolumn{1}{c|}{Pre.} & \multicolumn{1}{c|}{Recall} & \multicolumn{1}{c|}{AUC} \\ \hline
Res. time         & 0.38  & 0.03    & 0.67 & 0.12  & 0.46    & 0.54 & 0.53  & 0.09    & \textit{0.77}            \\ \hline
Cpu               & 0.66  & 0.06    & 0.73 & 0.27  & 0.29    & 0.61 & 0.48  & 0.14    & \textit{0.76}            \\ \hline
Memory            & 0.49  & 0.04    & 0.71 & 0.16  & 0.40    & 0.55 & 0.48  & 0.10    & \textit{0.73}            \\ \hline
I/O Read          & 0.70  & 0.05    & 0.71 & 0.22  & 0.33    & 0.57 & 0.46  & 0.18    & \textit{0.80}            \\ \hline
I/O Write         & 0.50  & 0.06    & 0.64 & 0.33  & 0.22    & 0.57 & 0.50  & 0.14    & \textit{0.66}            \\ \hline
Average           & 0.55  & 0.05    & 0.69 & 0.22  & 0.34    & 0.57 & 0.49  & 0.13    & \textit{0.74}            \\ \hline
\multirow{2}{*}{} & \multicolumn{3}{c|}{XG with tf-idf}        & \multicolumn{3}{c|}{XG with PCA}           & \multicolumn{3}{c|}{XG with code embedding}\\ \cline{2-10} 
                  & \multicolumn{1}{c|}{Pre.} & \multicolumn{1}{c|}{Recall} & \multicolumn{1}{c|}{AUC} & \multicolumn{1}{c|}{Pre.} & \multicolumn{1}{c|}{Recall} & \multicolumn{1}{c|}{AUC} & \multicolumn{1}{c|}{Pre.} & \multicolumn{1}{c|}{Recall} & \multicolumn{1}{c|}{AUC} \\ \hline
Res. time         & 0.65  & 0.42    & \textit{\textbf{0.94}}   & 1.00  & 0.05    & 0.60 & 0.71  & 0.31    & 0.93 \\ \hline
Cpu               & 0.66  & 0.48    & \textit{0.88}            & 0.32  & 0.06    & 0.62 & 0.67  & 0.50    & \textit{0.88}            \\ \hline
Memory            & 0.66  & 0.32    & \textit{0.87}            & 0.41  & 0.04    & 0.68 & 0.72  & 0.32    & \textit{0.87}            \\ \hline
I/O Read          & 0.66  & 0.49    & \textit{0.91}            & 0.49  & 0.08    & 0.73 & 0.73  & 0.50    & \textit{0.91}            \\ \hline
I/O Write         & 0.66  & 0.38    & \textit{0.82}            & 0.41  & 0.16    & 0.58 & 0.67  & 0.40    & 0.80 \\ \hline
Average           & 0.66  & 0.42    & \textit{0.88}            & 0.52  & 0.08    & 0.64 & 0.70  & 0.41    & 0.88 \\ \hline
\multirow{2}{*}{} & \multicolumn{3}{c|}{NN with tf-idf}        & \multicolumn{3}{c|}{NN with PCA}           & \multicolumn{3}{c|}{NN with code embedding}\\ \cline{2-10} 
                  & \multicolumn{1}{c|}{Pre.} & \multicolumn{1}{c|}{Recall} & \multicolumn{1}{c|}{AUC} & \multicolumn{1}{c|}{Pre.} & \multicolumn{1}{c|}{Recall} & \multicolumn{1}{c|}{AUC} & \multicolumn{1}{c|}{Pre.} & \multicolumn{1}{c|}{Recall} & \multicolumn{1}{c|}{AUC} \\ \hline
Res. time         & 0.34  & 0.54    & 0.79 & 0.27  & 0.83    & \textit{0.80}            & 0.27  & 0.64    & 0.75 \\ \hline
Cpu               & 0.53  & 0.30    & 0.72 & 0.63  & 0.41    & \textit{0.73}            & 0.39  & 0.33    & 0.65 \\ \hline
Memory            & 0.43  & 0.27    & \textit{0.67}            & 0.52  & 0.34    & \textit{0.67}            & 0.31  & 0.42    & 0.66 \\ \hline
I/O Read          & 0.53  & 0.44    & 0.73 & 0.60  & 0.46    & \textit{0.76}            & 0.48  & 0.33    & 0.72 \\ \hline
I/O Write         & 0.50  & 0.38    & \textit{0.68}            & 0.53  & 0.32    & 0.65 & 0.39  & 0.41    & \textit{0.68}            \\ \hline
Average           & 0.47  & 0.39    & \textit{0.72}            & 0.51  & 0.47    & \textit{0.72}            & 0.37  & 0.43    & 0.69 \\ \hline
\multirow{2}{*}{} & \multicolumn{3}{c|}{CNN with tf-idf}       & \multicolumn{3}{c|}{CNN with PCA}          & \multicolumn{3}{c|}{CNN with code embedding}                   \\ \cline{2-10} 
                  & \multicolumn{1}{c|}{Pre.} & \multicolumn{1}{c|}{Recall} & \multicolumn{1}{c|}{AUC} & \multicolumn{1}{c|}{Pre.} & \multicolumn{1}{c|}{Recall} & \multicolumn{1}{c|}{AUC} & \multicolumn{1}{c|}{Pre.} & \multicolumn{1}{c|}{Recall} & \multicolumn{1}{c|}{AUC} \\ \hline
Res. time         & 0.29  & 0.48    & 0.75 & 0.06  & 0.90    & 0.73 & 0.23  & 0.51    & \textit{0.79}            \\ \hline
Cpu               & 0.22  & 0.68    & 0.78 & 0.18  & 0.78    & 0.76 & 0.63  & 0.25    & \textit{0.81}            \\ \hline
Memory            & 0.47  & 0.25    & 0.69 & 0.13  & 0.87    & 0.74 & 0.20  & 0.57    & \textit{0.76}            \\ \hline
I/O Read          & 0.32  & 0.41    & \textit{0.68}            & 0.27  & 0.25    & \textit{0.68}            & 0.20  & 0.38    & 0.66 \\ \hline
I/O Write         & 0.27  & 0.31    & 0.64 & 0.14  & 0.64    & 0.65 & 0.19  & 0.60    & \textit{0.67}            \\ \hline
Average           & 0.31  & 0.43    & 0.71 & 0.16  & 0.69    & 0.71 & 0.29  & 0.46    & \textit{0.74}            \\ \hline
\end{tabular}

\label{tab:model_evaluation_hadoop}
\end{table}




\begin{table}
\tabcolsep=0.18cm
\caption{\emph{Cassandra}'s results of using different models to predict whether configuration options cause the manifesting of performance regression. The best results for each performance metric and each model are highlighted in \textit{italic}. The best results for each performance metric across different models are highlighted in \textbf{\textit{bold-italic}}.}
\begin{tabular}{|c|r|r|r|r|r|r|r|r|r|}
\hline
\multicolumn{10}{|c|}{Cassandra}          \\ \hline
\multirow{2}{*}{} & \multicolumn{3}{c|}{RF with tf-idf}      & \multicolumn{3}{c|}{RF with PCA}         & \multicolumn{3}{c|}{RF with code embedding}                   \\ \cline{2-10} 
                  & \multicolumn{1}{c|}{Pre.} & \multicolumn{1}{c|}{Recall} & \multicolumn{1}{c|}{AUC} & \multicolumn{1}{c|}{Pre.} & \multicolumn{1}{c|}{Recall} & \multicolumn{1}{c|}{AUC} & \multicolumn{1}{c|}{Pre.} & \multicolumn{1}{c|}{Recall} & \multicolumn{1}{c|}{AUC} \\ \hline
Res. time         & 0.74 & 0.37   & 0.74& 0.45 & 0.13   & 0.62& 0.67 & 0.46   & \textit{0.75}            \\ \hline
Cpu               & 0.68 & 0.39   & 0.76& 0.46 & 0.15   & 0.61& 0.73 & 0.59   & \textit{0.82}            \\ \hline
Memory            & 0.71 & 0.37   & 0.78& 0.35 & 0.04   & 0.61& 0.71 & 0.58   & \textit{\textbf{0.84}}   \\ \hline
I/O Read          & 0.74 & 0.48   & 0.79& 0.54 & 0.32   & 0.67& 0.74 & 0.63   & \textit{\textbf{0.83}}   \\ \hline
I/O Write         & 0.76 & 0.50   & 0.82& 0.58 & 0.32   & 0.68& 0.77 & 0.65   & \textit{\textbf{0.86}}   \\ \hline
Average           & 0.73 & 0.42   & 0.78& 0.47 & 0.19   & 0.64& 0.72 & 0.58   & \textit{0.82}            \\ \hline
\multirow{2}{*}{} & \multicolumn{3}{c|}{LR with tf-idf}      & \multicolumn{3}{c|}{LR with PCA}         & \multicolumn{3}{c|}{LR with code embedding}                   \\ \cline{2-10} 
                  & \multicolumn{1}{c|}{Pre.} & \multicolumn{1}{c|}{Recall} & \multicolumn{1}{c|}{AUC} & \multicolumn{1}{c|}{Pre.} & \multicolumn{1}{c|}{Recall} & \multicolumn{1}{c|}{AUC} & \multicolumn{1}{c|}{Pre.} & \multicolumn{1}{c|}{Recall} & \multicolumn{1}{c|}{AUC} \\ \hline
Res. time         & 0.38 & 0.38   & 0.59& 0.28 & 0.54   & 0.54& 0.38 & 0.51   & \textit{0.63}            \\ \hline
Cpu               & 0.49 & 0.42   & 0.64& 0.33 & 0.41   & 0.53& 0.46 & 0.58   & \textit{0.65}            \\ \hline
Memory            & 0.44 & 0.26   & 0.62& 0.29 & 0.28   & 0.55& 0.43 & 0.47   & \textit{0.66}            \\ \hline
I/O Read          & 0.50 & 0.50   & 0.63& 0.35 & 0.44   & 0.55& 0.49 & 0.61   & \textit{0.67}            \\ \hline
I/O Write         & 0.53 & 0.51   & \textit{0.69}            & 0.36 & 0.37   & 0.54& 0.47 & 0.63   & 0.68\\ \hline
Average           & 0.47 & 0.41   & 0.64& 0.32 & 0.41   & 0.54& 0.44 & 0.56   & \textit{0.66}            \\ \hline
\multirow{2}{*}{} & \multicolumn{3}{c|}{XG with tf-idf}      & \multicolumn{3}{c|}{XG with PCA}         & \multicolumn{3}{c|}{XG with code embedding}                   \\ \cline{2-10} 
                  & \multicolumn{1}{c|}{Pre.} & \multicolumn{1}{c|}{Recall} & \multicolumn{1}{c|}{AUC} & \multicolumn{1}{c|}{Pre.} & \multicolumn{1}{c|}{Recall} & \multicolumn{1}{c|}{AUC} & \multicolumn{1}{c|}{Pre.} & \multicolumn{1}{c|}{Recall} & \multicolumn{1}{c|}{AUC} \\ \hline
Res. time         & 0.66 & 0.38   & \textit{0.75}            & 0.37 & 0.20   & 0.60& 0.63 & 0.44   & 0.74\\ \hline
Cpu               & 0.65 & 0.49   & 0.77& 0.49 & 0.32   & 0.63& 0.70 & 0.60   & \textit{0.82}            \\ \hline
Memory            & 0.65 & 0.49   & 0.80& 0.45 & 0.13   & 0.60& 0.70 & 0.56   & \textit{0.83}            \\ \hline
I/O Read          & 0.69 & 0.55   & 0.79& 0.46 & 0.35   & 0.61& 0.72 & 0.62   & \textit{0.81}            \\ \hline
I/O Write         & 0.74 & 0.59   & 0.84& 0.48 & 0.33   & 0.65& 0.74 & 0.64   & \textit{0.85}            \\ \hline
Average           & 0.68 & 0.50   & 0.79& 0.45 & 0.27   & 0.62& 0.70 & 0.57   & \textit{0.81}            \\ \hline
\multirow{2}{*}{} & \multicolumn{3}{c|}{NN with tf-idf}      & \multicolumn{3}{c|}{NN with PCA}         & \multicolumn{3}{c|}{NN with code embedding}                   \\ \cline{2-10} 
                  & \multicolumn{1}{c|}{Pre.} & \multicolumn{1}{c|}{Recall} & \multicolumn{1}{c|}{AUC} & \multicolumn{1}{c|}{Pre.} & \multicolumn{1}{c|}{Recall} & \multicolumn{1}{c|}{AUC} & \multicolumn{1}{c|}{Pre.} & \multicolumn{1}{c|}{Recall} & \multicolumn{1}{c|}{AUC} \\ \hline
Res. time         & 0.55 & 0.36   & 0.71& 0.53 & 0.40   & \textit{0.73}            & 0.47 & 0.41   & 0.67\\ \hline
Cpu               & 0.27 & 0.94   & 0.88& 0.31 & 0.96   & \textit{\textbf{0.90}}   & 0.26 & 0.88   & 0.84\\ \hline
Memory            & 0.56 & 0.51   & 0.76& 0.64 & 0.57   & \textit{\textbf{0.84}}   & 0.61 & 0.49   & 0.76\\ \hline
I/O Read          & 0.66 & 0.46   & 0.75& 0.67 & 0.57   & \textit{\textbf{0.83}}   & 0.60 & 0.54   & 0.74\\ \hline
I/O Write         & 0.67 & 0.39   & 0.77& 0.70 & 0.57   & \textit{0.84}            & 0.63 & 0.55   & 0.77\\ \hline
Average           & 0.54 & 0.53   & 0.77& 0.57 & 0.62   & \textit{\textbf{0.83}}   & 0.51 & 0.57   & 0.76\\ \hline
\multirow{2}{*}{} & \multicolumn{3}{c|}{CNN with tf-idf}     & \multicolumn{3}{c|}{CNN with PCA}        & \multicolumn{3}{c|}{CNN with code embedding}                  \\ \cline{2-10} 
                  & \multicolumn{1}{c|}{Pre.} & \multicolumn{1}{c|}{Recall} & \multicolumn{1}{c|}{AUC} & \multicolumn{1}{c|}{Pre.} & \multicolumn{1}{c|}{Recall} & \multicolumn{1}{c|}{AUC} & \multicolumn{1}{c|}{Pre.} & \multicolumn{1}{c|}{Recall} & \multicolumn{1}{c|}{AUC} \\ \hline
Res. time         & 0.30 & 0.28   & 0.75& 0.37 & 0.35   & \textit{\textbf{0.79}}   & 0.33 & 0.34   & 0.75\\ \hline
Cpu               & 0.29 & 0.37   & \textit{0.76}            & 0.25 & 0.67   & 0.75& 0.11 & 0.96   & \textit{0.76}            \\ \hline
Memory            & 0.37 & 0.21   & \textit{0.77}            & 0.37 & 0.21   & \textit{0.77}            & 0.33 & 0.30   & 0.74\\ \hline
I/O Read          & 0.19 & 0.53   & \textit{0.69}            & 0.30 & 0.37   & 0.68& 0.24 & 0.47   & \textit{0.69}            \\ \hline
I/O Write         & 0.23 & 0.40   & \textit{0.70}            & 0.19 & 0.38   & 0.67& 0.23 & 0.33   & 0.69\\ \hline
Average           & 0.28 & 0.36   & \textit{0.73}            & 0.30 & 0.40   & \textit{0.73}            & 0.25 & 0.48   & \textit{0.73}            \\ \hline
\end{tabular}
\label{tab:model_evaluation_cassandra}
\end{table}

%\med{what about adding (1) a model that predicts when a \instance suffers from an \inconsistent on at least one performance measure and (3) another model on all the performance measures. the goal of this is to simulate different use cases. the first use case (1) is for people who are interested on any performance metric, what (2) we already have is for people who are interested on one precise performance metric, and the (3) is for people who are looking for very bad options that cause problems on all the performance metrics. } \jinfu{If practitioners want to predict whether a \instance suffers from \inconsistent in any of performance metric, practitioners can combine the five predictors. In other words, any of the model in five performance metrics predicts a \instance suffers from \inconsistent.}

\noindent \textbf{Our models can effectively predict when a \instance is manifesting an \inconsistent for all of our five studied performance measures,} as shown in Table~\ref{tab:model_evaluation_hadoop} and Table~\ref{tab:model_evaluation_cassandra}. % show the performance of our models for predicting whether adjusting a configuration option would cause difference in the results of performance regression detection. 
Our best models (i.e., as indicated by the \textbf{\textit{bold-italic}} values) achieve an AUC of 0.85 to 0.94 on the \emph{Hadoop} project and 0.79 to 0.90 on the \emph{Cassandra} project, for different performance metrics.
For the \emph{Hadoop} project, %~\heng{make the wording system or project consistent throughout the paper}, 
RF is the best model for four out of the five performance metrics, which achieves an AUC of 0.85 to 0.93. Even if XG shows the best AUC performance for the fifth performance metric (i.e., Response time), the difference between RF and XG is only 0.01. 
For the \emph{Cassandra} project, RF shows the best performance on three out of five performance metrics. NN shows the best performance on also three performance metrics (Memory and I/O read are the same to RF model). The average AUC of the best NN modle is 0.83 while the average AUC of the best RF model is 0.82. 
Note that %\med{is the following correct?} 
NN, on the other side, requires a large amount of resources to train and test a model, while the improvements it shows over RF is trivial.
CNN shows the best performance on only one performance metric (i.e., with an AUC of 0.79 for the Response time). However, the average AUC of the best CNN model is 0.09 lower than that of RF. 
%Note that NN and RF show similar AUC performance on two performance metrics. % the best models for different performance metrics. 
%In summary, most of the performance metrics can be effectively predicted by RF, while other metrics show a low AUC performance differences with other models. 
In summary, we suggest that developers consider the RF model for predicting when a \instance has a \inconsistent problem. 
%, a precision of 0.72 to 0.79, a recall of 0.33 to 0.60 for the Hadoop project, and an AUC of 0.75 to 0.86, a precision of 0.67 to 0.77, a recall of 0.46 to 0.65 for the Cassandra project.
%an average \emph{AUC} of 0.83 and 0.75 in \emph{Hadoop} and \emph{Cassandra}, respectively. 

%\med{I don't see any value on this comparison. both of them are good}\noindent\textbf{Our models are better at distinguishing the \instance cases that need adjustments of the configuration options and that do not for the Hadoop project than for the Cassandra project.} 
%Table~\ref{tab:model_evaluation_hadoop} and Table~\ref{tab:model_evaluation_cassandra} show that 
Our best models achieve a better performance for the \emph{Hadoop} system 
%(with an AUC of 0.85 to 0.94) 
than for the \emph{Cassandra} system. % (with an AUC of 0.79 to 0.90).  %for the Hadoop project (best average AUC is 0.91) and a lower AUC of 0.84 to 0.90 for the Cassandra project (best average AUC is 0.83). 
As discussed in RQ1, the different commits show more inconsistent $<test, option, \inconsistent>$ triplets (i.e., more dark cells in Figure~\ref{fig:across-commit-cassandra}) in the \emph{Cassandra} system than in the \emph{Hadoop} system, thus it is more difficult to predict \inconsistent for the \emph{Cassandra} system, which could explain the reason that our models perform better for the \emph{Hadoop} project.
%In particular, for the Hadoop project, our models achieve the best performance for predicting the impact of adjusting configurations on the detection of response time regressions (with best AUC of 0.97) and worst performance on I/O write regressions (with best AUC of 0.85). For the Cassandra project, our models achieve more consistent performance across different performance metrics.

\noindent \textbf{Using different representations of the code tokens significantly impact the performance of our models.} As shown in Table~\ref{tab:model_evaluation_hadoop} and Table~\ref{tab:model_evaluation_cassandra}, for the traditional models (RF, LR, and XG), using code embeddings to represent the code tokens often achieves the best performance, while using PCA usually results in the worst performance. For example, for the \emph{Hadoop} project, the RF model achieves an AUC of 0.85 to 0.93 using code embeddings, 0.82 to 0.93 using tf-idf, and only 0.59 to 0.76 using PCA. The reason for the poor performance of the models using PCA might be that PCA significantly reduced the information in the tokens through dimension reduction, even though we considered the principal components that account for 95\% of the variance in the original variables. 
In contrast, for the deep neural network models (NN and CNN), using PCA to represent the code tokens may achieve better results than the other two representations. For example, for the \emph{Cassandra} project, the CNN model combined with PCA achieves the best AUC for two out of the five performance metrics, across all different models.
The reason might be that there are much more options in the deep neural network models, while using PCA could could significantly reduce the number of options to be trained.

%\noindent \textbf{Our models can be used to equally predict performance regression introduced by adjusting configuration option in all performance metrics.} By examining the prediction results with all performance metrics, we find that all the performance metrics have similar \emph{AUC}s. 
%\noindent \textbf{For the Hadoop project, our models achieve the best performance for predicting the impact of adjusting configurations on the detection of response time regressions and worst performance on I/O write regressions. For the Cassandra project, our models achieve more consistent performance across different performance metrics.} As shown in Table~\ref{tab:model_evaluation_hadoop} and Table~\ref{tab:model_evaluation_cassandra}, for the Hadoop project, our best models achieve an AUC of 0.97 for the response time metric and an AUC of 0.85 for the I/O write metric. 
%In comparison, our best models achieve an AUC of 0.84 to 0.90 for the different performance metrics of the Cassandra project.
%our RF model with code embeddings achieves an AUC of 0.93 on reponse time regressions and an AUC of 0.85 on I/O write regressions. On the contrary, the same model achieves an AUC of 0.74 on response time regressions and an AUC of 0.85 on I/O write regressions.
%The difference can be resulted from the different characteristics of the two systems: as a distributed computing platform, the response time of Hadoop is sensitive to many factors including the configuration options; as a database system, there are frequent I/O write in the Cassandra system, thus the I/O write performance could be significantly impacted by the configuration options.

%\heng{Using tree-based models (RF and XGBoost together with TF-IDF or code embeddings for encoding the code tokens achieves the best performance for predicting whether a configuration will have a significant impact on the performance testing results.}

%\heng{Add a discussion referring to the heatmaps in Figure 3 and Figure 5 to discuss why Hadoop gets better performance than Cassandra: the impact of the configuration options are more consistent in Hadoop.}


\vspace{0.5cm}
\begin{Summary}{Summary of RQ1}{}
Our models can effectively predict whether a \instance manifests an \inconsistent problem. Random forest based on code embedding shows the best performance on predicting \inconsistent for most of the performance measures.
\end{Summary}


\section{Threats to Validity}
\label{sec:threats}
In this section, we discuss the threats to the validity of our study. 

\noindent \textbf{External validity.} The first external threat to validity concerns the generalizability of our results to other software systems. Due to the expensive computing resources needed (we spent around 12,536 machine hours collecting performance data), we conducted our evaluation on two open-source software systems, i.e., \emph{Hadoop} and \emph{Cassandra}. Our findings may not generalize to other software systems. %Our evaluated software systems are implemented in \emph{Java}. Therefore, our findings might not be generalizable to software systems implemented in other programming languages.
However, we found motivating results on the prevalence of \inconsistent and the performance of our prediction model, which can be replicated by future studies on other software systems. % can evaluate our approach and findings on commercial software systems with other languages. 

\noindent \textbf{Internal validity.}
When we collect our commits data, We only consider the last commit, if multiple commits are associated with the same issue. We may miss some commits with code changes. Future work can study the difference if considering all the commits.
Another internal threat to validity concerns the performance metrics that we consider. In our approach, we collect five popular performance metrics, i.e., Response time, CPU, Memory, I/O read and write, while other performance metrics such as throughput can still be explored by future research.% can choose other performance metrics to complement our study. 
We do not consider the combination of configuration options as that will require a huge cost and because the goal of our study is to identify and define the \inconsistent issues. On the one hand, a prior work~\cite{RN2864} mentions that 72\% of the performance issues are due to a single option, so our paper is covering the most common cause of performance issues. On the other hand, one can use covering arrays to conduct N-way testing for functional tests but with very low number of cases~\cite{colbourn2004combinatorial}. However, for performance testing, the approach likely does not work since we want to isolate each combination's performance impact from others. We encourage future studies to extend our work by considering the interaction of configuration options.

Similarly, our prediction model does not cover all the possible dimensions of metrics. For example, we do not consider a developer dimension. However, our model shows a good AUC performance to predict whether a \instance manifests an \inconsistent. Future studies can explore more dimensions of metrics to improve the performance of our models. 

Finally, our evaluation considers just the traditional models (i.e., logistic regression, random forest, and XGBoost) and neural network models (i.e., general neural network and convolutional neural network). Although we do not cover all existing models, our study covers the most popular ones that are used in software engineering. Future work is encouraged to explore more models. 


%In RQ2, we use four dimension metrics, i.e., code change, code structure, code token, and configuration option to build our prediction model. However, there exist other dimension metrics such as metrics from developer.  

%We choose three traditional models, i.e., logistic regression, random forest, and XGBoost and two neural network models, i.e., general neural network and convolutional neural network to construct our predictors. There are more machine learning algorithms can be used to build our model. Future research can add more metrics and use other machine learning algorithms to improve the results.

\noindent \textbf{Construct validity.} %Our evaluation is based on the stability and correctness of collected performance data. The quality and correctness of recorded performance data can impact our study. We assume that the performance monitoring tool named \emph{psutil} can accurately capture performance metrics. \emph{psutil} has been widely used in software performance study~\cite{DBLP:conf/icsm/AlghmadiSSH16, DBLP:conf/icsm/ChenS17}. 
% There might exist environmental performance noise when we use \emph{psutil} to capture the performance data. To minimize the noise, we capture the performance of the corresponding Linux process of the running tests. Furthermore, for each test, we repeat the execution 30 times independently. %\med{is this correct?}
% Finally, we run all of our experiments in the same environments. %\bram{what does ``configuration'' mean here?}. 
% \jinfu{ }
The stability of the cloud-based testing environment may cause testing noise and pose a threat to the validity of the measured performance data. To minimize the noise, we capture the performance of the corresponding Linux process of the running tests. Furthermore, for each test, we repeat the execution 30 times independently. Finally, we run all of our experiments in the same environments. There may still exist extreme values as outliers that should not be considered by our approach. To mitigate this threat, we remove the outlier data using the mean ± 3 × standard deviation (STD) as an indicator of outliers.

\section{Related Work}
\label{sec:relatedwork}

In this section, we discuss prior work along three dimensions: software performance regression detection, performance model for configurable system, and identifying optimal configuration for performance. % engineering.

\subsection{Performance regression detection}
Extensive prior research has proposed automated techniques to detect performance regressions. Such detection techniques can be divided into two categories: measurement-based and model-based detection. 

Measurement-based approaches compare performance metrics (e.g., CPU usage) between two consecutive versions to detect performance regressions. %measure performance metrics and compares these performance metrics between two consecutive versions of a system to detect performance regression. 
For example, Nguyen \emph{et al$.$}~\cite{Nguyen:2012:ADP,nguyen2011automated,Nguyen:2014:ICS} %conducted a series of studies on performance regressions. Nguyen \emph{et al$.$} apply control charts to analyze performance counters across test runs to detect performance regression automatically. They construct the control chart to detect performance regressions  by setting upper and lower bounds of performance counters.
leveraged control charts to identify performance regressions. %~\heng{treat regression as a countable word throughout the paper, countable seems better}. 
A control chart has an upper control limit and a lower control limit. A performance regression is detected when a performance metric is above the upper limit or below the lower limit. Foo \emph{et al$.$}~\cite{foo2010mining} proposed an approach that compares a test's performance metrics to historical performance metrics. %\med{from what?}performance regression testing repositories to detect potential performance regressions. 

A model-based approach builds %\med{This paragraph is difficult to understand. What is detected model? is it prediction model? and what is counters? what is signatures? what do you mean by heterogeneous environment?} 
%\jinfu{detected model is a general model, we can use model. Performance counter is performance metric, like CPU usage. Some papers use signatures to represent system users' behavior.} \med{I updated this paragraph, though Bodik work is not clear} 
a machine learning model with a set of performance metrics to detect performance regressions.  Cohen et al.~\cite{Cohen:2005:CIC} showed an implication that it is ineffective and not enough to index and identify performance problems with simple records of raw system metrics. Cohen et al. used TAN (Tree-Augmented Bayesian Network) models to model the system performance states based on a small subset of metrics. %Therefore, the authors present an approach to capture signatures representing system states from a running system and cluster such signitures to detect recurrent or similar performance problems. 
Bodik \emph{et al$.$}~\cite{bodik2008hilighter} leveraged a logistic regression model to model system users' behavior to improve Cohen \emph{et al$.$}\textquotesingle s model. %~\heng{need to mention Cohen's model beforehand}. 
Foo \emph{et al$.$}~\cite{DBLP:conf/icse/FooJAHZF15} proposed an approach that uses ensembles of models to detect performance regressions in heterogeneous environments (e.g., different hardware and software configurations). % \med{examples of what heterogeneous environments} \jinfu{different hardware and software configurations}). 
Xiong \emph{et al$.$}~\cite{Xiong:2013:VAM} proposed a model-driven framework to diagnose application performance and identify the root cause of performance issues. %Such framework uses linear regression to build the predict model to automatically diagnose the system performance in cloud environment and lead to root cause of performance problem.

%\med{well, we predict it, so it's kind of detecting performance regression}Our research is not designed to detect performance regression. The goal of our research is to examine the impact of configuration options on the performance regression. In our paper, we conduct measurement-based approach to identify performance regression on the commit level.

%\med{is the following paragraph correct?}
Our work complements this line of research in the sense that we consider the configuration aspect of highly configurable software systems. For instance, a code change might not show a performance regression on the default configuration, while leading to regressions on other configurations. This paper sheds light on the \inconsistent problem by first quantifying the existence of inconsistent performance variations, then proposing a prediction model that identifies the commits, tests, and options that exhibit the \inconsistent problem. 


\subsection{Performance model for configurable system}
Many prior research has been conducted on predicting performance for configurable software systems. 
M{\"{u}}hlbauer et al.~\cite{DBLP:conf/kbse/MuhlbauerAS19,DBLP:conf/kbse/MuhlbauerAS20} build performance model to identify performance changes in software performance evolution. The authors combine active learning and Gaussian process regression to model and estimate a software system’s performance evolution with only a few revision measurements. Such prior study provides an evidence that performance changes during software evolution, which motivates our study. Jamshidi et al.~\cite{DBLP:conf/kbse/JamshidiSVKPA17,DBLP:conf/icse/JamshidiVKSK17,DBLP:conf/sigsoft/JamshidiVKS18} employ transfer learning to learn performance model across environments. Work~\cite{DBLP:conf/kbse/JamshidiSVKPA17} conducts an exploratory study on four systems to investigate how a performance model for a configurable system changes when the system is deployed in different environments. The result shows that for severe environment changes, a considerable part of invalid configuration is preserved across environmental changes. Following the exploratory study, Jamshidi et al.~\cite{DBLP:conf/icse/JamshidiVKSK17} propose an approach to learn the performance model for a real system using other sources such as simulator with a few measurements at a lower cost. The results show that transfer learning can achieve high prediction accuracy with only a few samples from the real-world system. Jamshidi et al.~\cite{DBLP:conf/sigsoft/JamshidiVKS18} propose learning to sample (L2S) that extract transferable knowledge from the source environment to help sample the configurations in the target environment. DeepPerf~\cite{DBLP:conf/icse/HaZ19} uses a deep feed-forward neural network to model configurable software system. Results show that DeepPerf can predict a system's performance with a small sample of configuration. 

The existing studies use the historical revisions’ performance, or the sampled configuration’s performance to build a model, to estimate the performance of future revision. Different from prior study, we measure all the configuration options and collect real performance data. In addition, we aim to find the inconsistent option performance variations during software evolution. In general, our study is orthogonal to the above approaches and our measured performance data can be used in future research on performance.

\subsection{Identifying optimal configuration for performance} 
A large body of research has been conducted on performance optimization by finding optimal configuration. 
Siegmund et al.~\cite{RN2880} build mathematical models that describe the impact of a configuration on software performance based on each option's value. Raghavachari et al.~\cite{RN3537} propose an iterative approach to identify an optimal configuration in terms of performance. Their approach consists of selecting for a J2EE web application a first configuration, compare its performance to a second configuration until the optimal configuration is found. Similarly, Dia et al.~\cite{RN3543} propose an approach that automatically adjusts the values of existing configuration options at run-time to optimize the CPU and memory usage objectives.
Li et al.~\cite{LiAutoConfig} leveraged performance monitoring data and execution logs to dynamically optimize the values of performance-related configuration options according to varying workloads in the field. 
Guo et al.~\cite{RN3544} leverage non-linear regression to suggest an optimal configuration.
It was extended to DECART~\cite{DBLP:journals/ese/GuoYSASVCWY18} that combines the CART with automated resampling and parameter tuning to predict performance of configurable systems. However, collecting a large amount of data for training a model that predicts the performance of a configuration is expensive.
Therefore, Sarkar et al.~\cite{RN3089} evaluated the progressive and projective sampling to train a model that predicts the performance of configuration. For their initial training sample, they consider data on which each option is enabled at least once. 
Nair et al.~\cite{DBLP:conf/sigsoft/NairMSA17,DBLP:journals/tse/Nair0MSA20} conduct several studies to find good performing configurations. 
Nair et al.~\cite{DBLP:conf/sigsoft/NairMSA17} first propose a rank-based approach that uses the performance model building from a few sample configurations, to rank the configurations, in order to find the faster configurations. 
Nair et al.~\cite{DBLP:journals/tse/Nair0MSA20} also propose a novel approach named FLASH that use a sequential model-based approach to find better configurations for a software system. 
Oh et al.~\cite{DBLP:conf/sigsoft/OhBMS17} propose a true random sampling to search configurations recursively to find near-optimal configurations without building a performance model.
Other efforts identified the optimal configuration options in terms of performance by leveraging existing optimization approaches, i.e., iterative search~\cite{RN3545}, multi-objective optimization~\cite{singh2016optimizing}, and smart hill climbing~\cite{RN3518}.

The goal of our paper is not to identify optimal configuration options or to debug an existing performance-related configuration error. Instead, we focus on studying inconsistent option performance through different commits. In particular, we focus on understanding whether a performance improvement or regression is consistent through all the values of an option. That is important, as one can improve the performance of a software system or release new changes that do not impact the performance under one configuration when other configurations hide a performance regression. 

Furthermore, prior work on this line of research compares the absolute performance between two values for the same option, while this can be subjective, as discussed earlier. One option's value can naturally consume performance as it enables the execution of additional features. However, performance comparison need also consider historical performance data. For instance, the execution of the software system under the same option's value can be improved %\bram{for those features? flow in this paragraph is unclear} 
compared the same option and value prior to that commit. In addition, a better performing option's value can show a regression compared to the prior commit as well. %impact of configuration options on the performance regression.% While a new commit does not show any performance regression under the default configuration, other configurations might show a low performance when the same configuration showed a good performance in the prior commit. 

%\subsection{Software Configuration}

%A large body of research efforts have been conducted on software configuration, which mainly focus on understanding configuration problems, preventing configuration errors, and debugging configuration errors. In this Section, we focus on the efforts that are related to the performance of configuration, while we refer to our prior systematic literature review~\cite{tse} for more details. %Few research efforts consider the performance aspect of software configuration \bram{wasn't there a lot of work on performance tuning in Mohammed's TSE survey?}. 

%\subsubsection{Understanding Configuration Problems}
%Configuration makes a software system complex~\cite{tse}, which leads to configuration errors that are severe, common, and hard to debug~\cite{RN3251}. For instance, Jin et al.~\cite{RN2897} found that configuration options add more complexity to the development and testing of highly configurable software systems. Han et al.~\cite{RN2864} found that configuration options are responsible for 59\% of the performance bugs. Gousios et al.~\cite{RN3551} observed that the configuration of the garbage collectors have an impact on the performance of server applications. Furthermore, Sayagh et al.~\cite{RN3249, RN2758} found that the impact of a configuration option can spread to multiple layers of an architectural stack.% Jin et al.~\cite{RN2897} found that there is a need for tools that debug configuration errors in multi-languages software systems. 

%Our work is different from this line of research as we consider the performance regression that is caused by configuration options.
%\subsubsection{Debugging Configuration Errors} 
%A second line of research proposed and evaluated different approaches to identify misconfigured configuration options. Dong et al.\cite{RN2805, RN3163} leverage the slicing technique to identify the misconfigured option for a given error message or exception. Rabkin et al.~\cite{RN2822} leverage a data flow analysis technique to identify for each option, which source code lines it might impacts. Attaryian et al.~\cite{RN3248} combined dynamic control and data flow analysis to identify misconfigured options. Zhang et al.~\cite{RN2839, RN2777} compared the trace of a correct execution against the trace of an incorrect execution to identify culprit options. We refer to our prior systematic literature review~\cite{tse} and the work of Tianyin et al.~\cite{RN3252} and Andrzejak et al.\cite{andrzejaksoftware} for further details about the existing configuration debugging approaches. 

%Our work is different from this line of research since we do not consider debugging configuration errors, but understanding and identifying performance regressions that are caused under certain configurations.

%\subsubsection{The Performance of Configurations} 

\begin{comment}
\subsection{Configuration-related performance bug}
Another line of research considers the debugging of performance errors that are caused by configuration options.
Attariyan et al.~\cite{RN3253} proposed an approach based on dynamic taint analysis technique to identify the option that causes a performance error. % assigns to each source code block a cost, use a dynamic analysis techniques that instruments and runs a software system, then their approach identifies which blocks were executed and which options they depends on. 
\end{comment}

\begin{comment}
Another line of research considers the identification of the optimal configurations for a software system and the debugging of performance errors that are caused by configuration options. 


Attariyan et al.~\cite{RN3253} proposed an approach based on dynamic taint analysis technique to identify the option that causes a performance error. % assigns to each source code block a cost, use a dynamic analysis techniques that instruments and runs a software system, then their approach identifies which blocks were executed and which options they depends on. 
Siegmund et al.~\cite{RN2880} build mathematical models that describe the impact of a configuration on software performance based on each option's value. Raghavachari et al.~\cite{RN3537} propose an iterative approach to identify an optimal configuration in terms of performance. Their approach consists of selecting for a J2EE web application a first configuration, compare its performance to a second configuration until the optimal configuration is found. Similarly, Dia et al.~\cite{RN3543} propose an approach that automatically adjusts the values of existing configuration options at run-time to optimize the CPU and memory usage objectives.

Li et al.~\cite{LiAutoConfig} leveraged performance monitoring data and execution logs to dynamically optimize the values of performance-related configuration options according to varying workloads in the field. Guo et al.~\cite{RN3544} leverage non-linear regression to suggest an optimal configuration. However, collecting a large amount of data for training a model that predicts the performance of a configuration is expensive. Therefore, Sarkar et al.~\cite{RN3089} evaluated the progressive and projective sampling to train a model that predicts the performance of configuration. For their initial training sample, they consider data on which each option is enabled at least once. Other efforts identified the optimal configuration options in terms of performance by leveraging existing optimization approaches, i.e., iterative search~\cite{RN3545}, multi-objective optimization~\cite{singh2016optimizing}, and smart hill climbing~\cite{RN3518}.
\end{comment}



\begin{comment}

\subsection{Software Performance}

Performance is an important aspect of software quality. Extensive prior research has been conducted to study software performance. In this subsection, we summarize the empirical studies on  %performance to 
understanding software performance and the studies on %. We then present the related works of 
performance regression detection.

\subsubsection{Empirical Studies on Software Performance}
Several empirical studies have been conducted on the performance of software systems~\cite{ICSE2014:Huang,Jin:2012,MSR11:Zaman,MSR12:Zaman,DBLP:conf/kbse/HanYL18,Leitner2017ICPE}. For instance, Jin \emph{et al$.$}~\cite{Jin:2012} studied 109 real-world performance issues that are reported from five open source software systems and %. Based on the studied 109 performance bugs, the authors 
%developed an automated tool 
proposed an approach to detect performance issues. Zaman \emph{et al$.$}~\cite{MSR11:Zaman,MSR12:Zaman} conducted qualitative and quantitative studies on performance issues. They found that developers and users face problems in reproducing performance bugs %. More time is 
as they spend % 
a lot of time discussing performance bugs %than 
compared to other kinds of bugs (e.g., functional bugs). %\jinfu{For example, Firefox performance bugs take more time to discuss and fix.\med{I meant examples of other kinds of bugs}} 
Huang \emph{et al$.$}~\cite{ICSE2014:Huang} %studied real world performance issues. They 
proposed an approach to improve the efficiency of performance regression testing by leveraging a static analysis technique to estimate the risk of a given commit in introducing a performance regression. Han et al$.$~\cite{DBLP:conf/kbse/HanYL18} studied %y
300 bug reports from three large open source projects. The authors found that most of the performance bugs occur for specific combinations of data input and configurations. They also proposed a framework named \emph{PerfLearning} to extract such data input and configurations from bug reports to generate test frames. Leitner et al$.$~\cite{Leitner2017ICPE} %aim to understand the current state of art of performance testing. They conduct a study on 111 open-source java-based systems from GitHub 
investigated the state-of-the-practice related to performance tests. The authors found that performance tests form only a small portion of a test suite.
%and use a combination of quantitative and qualitative research methods to investigate the use of performance tests across five perspectives.

The vast amount of research on software performance signifies its importance and motivates our work. %Prior studies on performance typically are based on either limited performance issue reports or release of the software. 
Different from prior research, we evaluate software performance at the commit level and study performance regressions that are manifested under a subset of the possible configurations. % by configuration option.
In addition, our work is different from this line of research as we consider how to avoid performance regressions that are related to some configurations while being hidden by other configurations, instead of understanding the existing performance related issues. 

\subsubsection{Performance regression detection}
Extensive prior research has proposed automated techniques to detect performance regressions. Such detection techniques can be divided into two categories: measurement-based and model-based detection. 

Measurement-based approaches compare performance metrics (e.g., CPU usage) between two consecutive versions to detect performance regressions. %measure performance metrics and compares these performance metrics between two consecutive versions of a system to detect performance regression. 
For example, Nguyen \emph{et al$.$}~\cite{Nguyen:2012:ADP,nguyen2011automated,Nguyen:2014:ICS} %conducted a series of studies on performance regressions. Nguyen \emph{et al$.$} apply control charts to analyze performance counters across test runs to detect performance regression automatically. They construct the control chart to detect performance regressions  by setting upper and lower bounds of performance counters.
leveraged control charts to identify performance regressions. %~\heng{treat regression as a countable word throughout the paper, countable seems better}. 
A control chart has an upper control limit and a lower control limit. A performance regression is detected when a performance metric is above the upper limit or below the lower limit. Foo \emph{et al$.$}~\cite{foo2010mining} proposed an approach that compares a test's performance metrics to historical performance metrics. %\med{from what?}performance regression testing repositories to detect potential performance regressions. 

A model-based approach builds %\med{This paragraph is difficult to understand. What is detected model? is it prediction model? and what is counters? what is signatures? what do you mean by heterogeneous environment?} 
%\jinfu{detected model is a general model, we can use model. Performance counter is performance metric, like CPU usage. Some papers use signatures to represent system users' behavior.} \med{I updated this paragraph, though Bodik work is not clear} 
a machine learning model with a set of performance metrics to detect performance regressions.  Cohen et al.~\cite{Cohen:2005:CIC} showed an implication that it is ineffective and not enough to index and identify performance problems with simple records of raw system metrics. Cohen et al. used TAN (Tree-Augmented Bayesian Network) models to model the system performance states based on a small subset of metrics. %Therefore, the authors present an approach to capture signatures representing system states from a running system and cluster such signitures to detect recurrent or similar performance problems. 
Bodik \emph{et al$.$}~\cite{bodik2008hilighter} leveraged a logistic regression model to model system users' behavior to improve Cohen \emph{et al$.$}\textquotesingle s model. %~\heng{need to mention Cohen's model beforehand}. 
Foo \emph{et al$.$}~\cite{DBLP:conf/icse/FooJAHZF15} proposed an approach that uses ensembles of models to detect performance regressions in heterogeneous environments (e.g., different hardware and software configurations). % \med{examples of what heterogeneous environments} \jinfu{different hardware and software configurations}). 
Xiong \emph{et al$.$}~\cite{Xiong:2013:VAM} proposed a model-driven framework to diagnose application performance and identify the root cause of performance issues. %Such framework uses linear regression to build the predict model to automatically diagnose the system performance in cloud environment and lead to root cause of performance problem.

%\med{well, we predict it, so it's kind of detecting performance regression}Our research is not designed to detect performance regression. The goal of our research is to examine the impact of configuration options on the performance regression. In our paper, we conduct measurement-based approach to identify performance regression on the commit level.

%\med{is the following paragraph correct?}
Our work complements this line of research in the sense that we consider the configuration aspect of highly configurable software systems. For instance, a code change might not show a performance regression on the default configuration, while leading to regressions on other configurations. This paper sheds light on the \inconsistent problem by first quantifying the existence of inconsistent performance variations, then proposing a prediction model that identifies the commits, tests, and options that exhibit the \inconsistent problem. 

\end{comment}
\input{tex/Conclusions}


\bibliographystyle{IEEEtranS} % from bare_jrnl_compsoc.tex
\small{
\bibliography{bibliography} 

% \vspace{8mm}



\begin{IEEEbiography}
[{\includegraphics[width=1in,height=1.25in,clip,keepaspectratio]{Bio-photo/Jinfu-Chen}}]
{Jinfu Chen}
is a Ph.D. student in the Department of Computer Science and Software Engineering at Concordia University, Montreal. He has received his M.Sc. degree from Chinese Academy of Sciences and he obtained B.Eng. from Harbin Institute of Technology. His research interest lies in empirical software engineering, software performance engineering, performance testing, software log mining. Contact him at \url{https://jinfuchen.github.io/jinfu}.
\end{IEEEbiography}

\begin{IEEEbiography}
[{\includegraphics[width=1in,height=1.25in,clip,keepaspectratio]{Bio-photo/Mohammed_Sayagh.jpg}}]
{Mohammed Sayagh}
is an assistant professor at ETS (Quebec University). Before that, he was a postdoctoral fellow at the Software Analysis and Intelligence Lab (SAIL) at Queen’s University. He obtained his PhD from the Lab on Maintenance, Construction, and Intelligence of Software (MCIS) at Ecole Polytechnique Montreal (Canada). His research interests include empirical software engineering, multi-component software systems, and software configuration engineering. More details about his work are available on \url{https://sailhome.cs.queensu.ca/~msayagh}.
\end{IEEEbiography}

\begin{IEEEbiography}
[{\includegraphics[width=1in,height=1.25in,clip,keepaspectratio]{Bio-photo/Heng-Li}}]
{Heng Li}
is an Assistant Professor in the Department of Computer Engineering and Software Engineering at Polytechnique Montreal, Montreal, Canada, where he leads the Maintenance, Operations and Observation of Software with intelligencE (MOOSE) lab. He obtained his Ph.D. from the School of Computing, Queen's University (Canada), M.Sc. from Fudan University (China), and B.Eng. from Sun Yat-sen University (China). He also worked at Synopsys as a software engineer for two years and worked at BlackBerry as a software performance engineer for another two years. His research interests lie within Software Engineering, in particular, software observability, intelligent operations of software systems, software log mining, software performance engineering, and mining software repositories. Contact him at: \url{heng.li@polymtl.ca}; \url{https://www.hengli.org}.
\end{IEEEbiography}

\begin{IEEEbiography}
[{\includegraphics[width=1in,height=1.25in,clip,keepaspectratio]{Bio-photo/bram}}]
{Bram Adams}
is an associate professor at Queen’s University. His research interests include mining software   repositories, software release   engineering   and   the   role   of   human   affect   in software   engineering.   His   work   has   been   published   at   premier   software   engineering   venues such   as   EMSE,   TSE,   ICSE,   FSE,   MSR,   ASE   and   ICSME.   In   addition   to   co-organizing   the RELENG International Workshop on Release Engineering from 2013 to 2015 (and the 1st IEEE Software Special Issue on Release Engineering), he co-organized the SEMLA, PLATE, ACP4IS, MUD and MISS workshops, and the MSR Vision 2020 Summer School. He has been PC co-chair of SCAM 2013, SANER 2015, ICSME 2016 and MSR 2019.
\end{IEEEbiography}

\begin{IEEEbiography}
[{\includegraphics[width=1in,height=1.25in,clip,keepaspectratio]{Bio-photo/Weiyi-Shang}}]
{Weiyi Shang}
is an Associate Professor and Concordia University Research Chair in Ultra-large-scale Systems at the Department of Computer Science and Software Engineering at Concordia University, Montreal. He has received his Ph.D. and M.Sc. degrees from Queens University (Canada) and he obtained B.Eng. from Harbin Institute of Technology. His research interests include big data software engineering, software engineering for Ultra-large-scale systems, software log mining, empirical software engineering, and software performance engineering. His work has been published at premier venues such as ICSE, FSE, ASE, ICSME, MSR and WCRE, as well as in major journals such as TSE, EMSE, JSS, JSEP and SCP. His work has won premium awards, such as two SIGSOFT Distinguished paper award at ICSE 2020 and 2013. His industrial experience includes helping improve the quality and performance of ultra-large-scale systems in BlackBerry. Early tools and techniques developed by him are already integrated into products used by millions of users worldwide. Contact him at \url{http://users.encs.concordia.ca/shang}.
\end{IEEEbiography}

}

\end{document}
