
\section{Background}
\label{sec:back}


%\subsection{Configuration}

Software configuration is a mechanism used to customize the behaviour of a software system without changing the source code. The configuration \textbf{options} are often stored in configuration files as a set of key - value pairs, where the key represents an option's name and the \textbf{value} represents a default or user-chosen value for that option. We define a \textbf{configuration} as one particular assignment of a value to all existing options. Table~\ref{tab:terms} lists the definition of these terms. For example, \emph{A=1} and \emph{B=2} is one possible configuration for a software system with the two integer options \emph{A} and \emph{B}. Configuration options enable users to adapt the execution of their software systems by simply modifying the values of certain configuration options, without re-compilation. For example, a user can change the directory that stores the cache for \emph{Cassandra} by changing the value of the \textit{saved\_caches\_directory} configuration option.% Such configuration can be changed at run-time without changing or recompiling the source code of the whole software system.

\begin{table}[t]
    \centering
    \footnotesize
    \tabcolsep=0.05cm
    \caption{Our definition of configuration, option, and value}
    \begin{tabular}{|p{1.1cm}|p{5.8cm}|l|}
        \hline
        Term & Definition & Example \\
        \hline
        \textbf{Option}  & A typed, configurable item that allows users to set different values. & $A$ \\ \hline
        \textbf{Value} & A specific assignment of a value for an option. & $A = 1$ \\ \hline
        \textbf{Configu-ration} & An assignment of values to all options by a user. & $A = 1; B = 2$ \\ 
        \hline
    \end{tabular}
    \label{tab:terms}
\end{table}


Although configuration introduces large flexibility for users, considering all the possible configurations during testing is impossible. A software system with 10 boolean configuration options requires testing $2{^1^0}$ configurations. In fact, configuration problems are among the dominant %\bram{exaggeration (since one of citations is ours)?} 
problems in software engineering~\cite{tse,RN2897}.

In particular, a software system can suffer from what we refer to as \textbf{Inconsistent Option Performance Variation} (a.k.a, \inconsistent). This occurs when, for a given commit \emph{C}, the performance of a subset of an option's values evolved differently relative to their performance in the commit prior to \emph{C}. Considering the example in Figure \ref{fig:description}, when comparing the raw performance of the two option values \emph{V1} and \emph{V2} (Figure~\ref{fig:description-a}), we observe that \emph{V1} shows a better performance than \emph{V2}. However, that might not be problematic as \emph{V2} might just enable an extra feature, such as logging a transaction. In fact, Figure~\ref{fig:description-c} shows that even if \emph{V2} does not show any significant performance variation from the prior commit, \emph{V1} suffers from a performance regression.  
Similarly, in Figure~\ref{fig:description-d} the performance of \emph{V2} is improved compared to the prior commit, while that improvement does not manifest under option value \emph{V1}. The \inconsistent may directly affect the user experience, increase the resources cost of the system and lead to reputational repercussions. %\bram{here (and also in intro) we need to argue that this is a more important problem than the first one, but right now this is not entirely convincing/concrete enough} %That indicates the first type of \inconsistent, which consists of an inconsistent improvement across different values of the same option. 
A performance variation is calculated as the difference between the performance variation of each option's value after and before each commit, which is illustrated in Figure~\ref{fig:description} by ``$a - b$''.


%%% Local Variables:
%%% mode: latex
%%% TeX-master: "../main"
%%% End:
